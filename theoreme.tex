\subsection*{Theoreme der Feldtheorie}
\begin{center}
\begin{tabular}{ccc} \toprule
Bezeichnung & Symbol & Einheit \\ \midrule
Beobachtungspunkt & $\vec{r}$ & \\
Quellpunkt & $\vec{r}'$ & \\
Magnetisches Vektorpotential & $\vec{A}$ & \\
Magnetisches Skalarpotential & $\psi$ & \\
Elektrisches Vektorpotential & $\vec{F}$ & \\
Elektrisches Skalarpotential & $\phi$ & \\
\bottomrule
\end{tabular}
\end{center}
\begin{center}
\begin{minipage}[c]{0.4\linewidth}
\begin{tikzpicture}[scale=0.5]
	\filldraw[pattern=dots,draw=white] (0,0) circle (3.3);
\filldraw[color=white,draw=black] (0,0) circle (3);
\filldraw[pattern=dots,draw=black] (1,1) circle (1);
\draw node[fill=white] at (1,1) {$V_a$};
\draw node at (-0.5,-1.5) {$V_b$};

\draw[->] (-3,0)--(-2.25,0) node[below] {$\vec{n}$};
\draw[->] (1,0)--(1,-0.75) node[right] {$\vec{n}$};
\draw (-1,2.75) -- (0,2.25) node[fill=white] {$A$} -- (0.75,2.1);

\draw[->] (-2,1)--+(0.5,0.75);
\draw[->] (-1.8,0.8)--+(0.5,0.75);
\draw[->] (-1.6,0.6)--+(0.5,0.75);
\draw (-2.0,0.7) node[fill=white] {$\vec{J}(\vec{r})$};

%\filldraw [color=black!10] (0,-0.5) rectangle(3,1);
%\coordinate (center) at (1.5,1);
%\draw (center) -- ++(30:1.3) ;
%\draw [->] (center) -- ++(150:1.3);
%\draw [dashed] (center) -- ++(0,1);
%\draw [dashed] (center) -- ++(0,-1.5);
%
%\draw (center) +(0,0.75)arc [radius=0.75,start angle=90, end angle=30];
%\draw (center) +(0,0.75)arc [radius=0.75,start angle=90, end angle=150];
%
%\path (center) ++(75:1) node{$\theta_1$};
%\path (center) ++(105:1) node{$\theta_1$};
%
%\draw [->] (center) -- ++(250:1.3);
%\draw (center) +(0,-0.75)arc [radius=0.75,start angle=270, end angle=250];
%\path (center) ++(260:1) node{$\theta_2$};
%
%\path (center) ++(15:1.3) node{$\epsilon_{r1}$};
%\path (center) ++(345:1.3) node{$\epsilon_{r2}$};
%\path (center) ++(0:1.2) node{$\wedge$};
\end{tikzpicture}
\end{minipage}
\begin{minipage}[c]{0.5\linewidth}
\paragraph{Eindeutigkeitstheorem} Normalenvektoren $\vec{n}$ zeigen immer ins Lösungsgebiet, elektrischer Strom $\vec{J}$ in/auf PEC strahlt kein Feld ab, magnetischer Strom $\vec{M}$ in/auf PMC strahlt kein Feld ab.
\end{minipage}
\end{center}

\begin{description}
\item[Magnetisches Vektorpotential]
\begin{align*}
\vec{B} &= \rot \vec{A} \\
\vec{E} &= -j \omega \vec{A} - \grad\phi
\intertext{mit Lorenz-Eichung}
\div\vec{A} &+ j\omega\mu\epsilon\phi = 0
\intertext{folgt die inhomogene, vektorielle Helmholtz-Gleichung}
\div\grad\vec{A} &+ k^2 \vec{A} = -\mu\vec{J}
\intertext{folgt die inhomogene, skalare Helmholtz-Gleichung}
\div\grad\phi &+ k^2 \phi = -\frac{\rho}{\epsilon}
\end{align*}

\item[Randbedingungen]
\begin{align*}
\text{PEC: }&\vec{n}\times\vec{E} = 0\\
\text{PMC: }&\vec{n}\times\vec{H} = 0
\end{align*}

\item[Elektrisches Vektorpotential]
\begin{align*}
\vec{D} &= -\rot F\\
\vec{H} &= -j \omega \vec{F} - \grad\psi
\intertext{mit Lorenz-Eichung}
\div\vec{F} &+ j\omega\mu\epsilon\psi = 0
\intertext{folgt die inhomogene, vektorielle Helmholtz-Gleichung}
\div\grad\vec{F} &+ k^2 \vec{F} = -\epsilon\vec{M}
\intertext{folgt die inhomogene, skalare Helmholtz-Gleichung}
\div\grad\psi &+ k^2 \psi = -\frac{\rho_m}{\mu}
\end{align*}

\item[Felddarstellung über Potentiale]
\begin{equation*}
\vec{E} (\vec{r}) = \vec{E}_e (\vec{r}) + \vec{E}_m (\vec{r}) \qquad \vec{H} (\vec{r}) = \vec{H}_e (\vec{r}) + \vec{H}_m (\vec{r})
\end{equation*}
\begin{equation*}
\begin{array}{c}
\toprule
\text{Elektrische Ströme} \\ 
\midrule
\vec{H}_e (\vec{r}) = \frac{1}{\mu} \rot{\vec{A}(\vec{r})} \\ 
\vec{E}_e (\vec{r}) = -j\omega \vec{A} -\grad{\phi} = -j\omega \left( \vec{A}(\vec{r}) + \frac{1}{k^2} \grad\div\vec{A}(\vec{r}) \right) \\
\bottomrule
\end{array}
\end{equation*}
\begin{equation*}
\begin{array}{c}
\toprule
\text{Magnetische Ströme} \\
\midrule
\vec{E}_m (\vec{r}) = - \frac{1}{\epsilon} \rot{\vec{F}(\vec{r})} \\ 
\vec{H}_m (\vec{r}) = -j\omega \vec{F} -\grad{\psi} = -j\omega \left( \vec{F}(\vec{r}) + \frac{1}{k^2} \grad\div\vec{F}(\vec{r}) \right) \\
\bottomrule
\end{array}
\end{equation*}

\item[Duale Größen] Felder elektrischer Quellen $\leftrightarrow$ Felder magnetischer Quellen
\begin{center}
\begin{tabular}{cc} \toprule
Elektrisch & Magnetisch \\ \midrule
$\vec{E_e}$ & $\vec{H_m}$ \\
$\vec{H_e}$ & $-\vec{E_m}$ \\
$\vec{A}$ & $\vec{F}$\\
$\phi$ & $\psi$ \\
$\vec{J}$ & $\vec{M}$ \\
$\rho$ & $\rho_m$\\
$\mu$ & $\epsilon$\\
$\epsilon$ & $\mu$\\
$k$ & $k$\\
$Z_F$ & $Z_F^{-1}$\\
\bottomrule
\end{tabular}
\end{center}

\item[Greensche Funktionen] Dirac-Impuls-förmige Anregung
%\begin{align*}
%G(\vec{r},\vec{r'}) &= \frac{\exp(-jk|\vec{r}-\vec{r'}|)}{4\pi|\vec{r}-\vec{r'}|} \\
%\overleftrightarrow{G}^A(\vec{r},\vec{r'}) &= \mu\frac{\exp(-jk|\vec{r}-\vec{r'}|)}{4\pi|\vec{r}-\vec{r'}|} \overleftrightarrow{I}\\
%\overleftrightarrow{G}^F(\vec{r},\vec{r'}) &= \epsilon\frac{\exp(-jk|\vec{r}-\vec{r'}|)}{4\pi|\vec{r}-\vec{r'}|} \overleftrightarrow{I}\\
%G^E_J(\vec{r},\vec{r'}) &= -j\omega\mu\left[\left( I + \frac{1}{k^2}\grad\grad \right) \frac{\exp(-jk|\vec{r}-\vec{r'}|)}{4\pi|\vec{r}-\vec{r'}|}\right]\\
%\overleftrightarrow{G}^H_J(\vec{r},\vec{r'}) &= \grad\frac{\exp(-jk|\vec{r}-\vec{r'}|)}{4\pi|\vec{r}-\vec{r'}|}\times \overleftrightarrow{I}\\
%\overleftrightarrow{G}^E_M(\vec{r},\vec{r'}) &= -\grad\frac{\exp(-jk|\vec{r}-\vec{r'}|)}{4\pi|\vec{r}-\vec{r'}|}\times \overleftrightarrow{I}\\
%\overleftrightarrow{G}^H_M(\vec{r},\vec{r'}) &= -j\omega\epsilon\left[\left( \overleftrightarrow{I} + \frac{1}{k^2}\grad\grad \right) \frac{\exp(-jk|\vec{r}-\vec{r'}|)}{4\pi|\vec{r}-\vec{r'}|}\right]\\
%\end{align*}
\begin{align*}
G(\vec{r},\vec{r'}) &= \frac{\exp(-jk|\vec{r}-\vec{r'}|)}{4\pi|\vec{r}-\vec{r'}|} \\
\overleftrightarrow{G}^A(\vec{r},\vec{r'}) &= \mu G(\vec{r},\vec{r'}) \overleftrightarrow{I}\\
\overleftrightarrow{G}^F(\vec{r},\vec{r'}) &= \epsilon G(\vec{r},\vec{r'}) \overleftrightarrow{I}\\
G^E_J(\vec{r},\vec{r'}) &= -j\omega\mu\left[\left( I + \frac{1}{k^2}\grad\div \right) G(\vec{r},\vec{r'}) \right]\\
\overleftrightarrow{G}^H_J(\vec{r},\vec{r'}) &= \grad G(\vec{r},\vec{r'}) \times \overleftrightarrow{I}\\
\overleftrightarrow{G}^E_M(\vec{r},\vec{r'}) &= -\grad G(\vec{r},\vec{r'}) \times \overleftrightarrow{I}\\
\overleftrightarrow{G}^H_M(\vec{r},\vec{r'}) &= -j\omega\epsilon\left[\left( \overleftrightarrow{I} + \frac{1}{k^2}\grad\div \right) G(\vec{r},\vec{r'}) \right]\\
\end{align*}
\begin{center}
{\begin{minipage}[t]{0.2\textwidth}
	\begin{tikzpicture}[every node/.style={scale=0.9}]
		\matrix (m) [
	matrix of nodes, 
    column sep=1cm,
    row sep=0.5cm,
    nodes={
    	draw=none,
    	},
    ]{
  	
  &
  	
  &
  	$\vec{E}_e$
  \\
  	
  &
  	$\overleftrightarrow{G}^E_J$
  &
  	
  \\
  	$\vec{J}$
  &
	$\overleftrightarrow{G}^A$
  &
  	$\vec{A}$
  \\
  	
  &
  	$\overleftrightarrow{G}^H_J$
  &
  	
  \\
  
  &

  &
  	$\vec{H}_e$
  \\
  };

{
	\path[line width=1pt] (m-3-1) edge (m-3-2);
	\path[line width=1pt,->] (m-3-2) edge (m-3-3);
	\path[line width=1pt] (m-3-1) edge (m-2-2);
	\path[line width=1pt,->] (m-2-2) edge (m-1-3);
	\path[line width=1pt] (m-3-1) edge (m-4-2);
	\path[line width=1pt,->] (m-4-2) edge (m-5-3);
	\path[line width=1pt,->] (m-3-3) edge (m-1-3);
	\path[line width=1pt,->] (m-3-3) edge (m-5-3);
    };
 
	\end{tikzpicture}
\end{minipage}
\begin{minipage}[t]{0.2\textwidth}
	\begin{tikzpicture}[every node/.style={scale=0.9}]
		\matrix (m) [
	matrix of nodes, 
    column sep=1cm,
    row sep=0.5cm,
    nodes={
    	draw=none,
    	},
    ]{
  	$\vec{E}_m$
  &
  	
  &
  	
  \\
  	
  &
  	$\overleftrightarrow{G}^E_M$
  &
  	
  \\
  	$\vec{F}$
  &
	$\overleftrightarrow{G}^F$
  &
  	$\vec{M}$
  \\
  	
  &
  	$\overleftrightarrow{G}^H_M$
  &
  	
  \\
  	$\vec{H}_m$
  &

  &
  	
  \\
  };

{
	\path[line width=1pt] (m-3-3) edge (m-3-2);
	\path[line width=1pt,->] (m-3-2) edge (m-3-1);
	\path[line width=1pt] (m-3-3) edge (m-2-2);
	\path[line width=1pt,->] (m-2-2) edge (m-1-1);
	\path[line width=1pt] (m-3-3) edge (m-4-2);
	\path[line width=1pt,->] (m-4-2) edge (m-5-1);
	\path[line width=1pt,->] (m-3-1) edge (m-1-1);
	\path[line width=1pt,->] (m-3-1) edge (m-5-1);
    };
 
	\end{tikzpicture}
\end{minipage}}
\end{center}

\item[Spiegelungsprinzip] An der Spiegelungsache löschen sich die entsprechenden Felder aus, um die Randbedingung zu erfüllen.
\begin{center}
\begin{tikzpicture}[scale=1]
	\def\breite{1.2};
\def\höhe{0.3};
\def\abstand{-1.5};
\filldraw[pattern=north east lines,draw=white] (0,-\höhe) rectangle +(\breite,\höhe) node[below,fill=white]{PEC};
\draw[very thick] (0,0) --+(\breite,0);
\filldraw[pattern=north east lines,draw=white] (0.5+\breite,-\höhe) rectangle +(\breite,\höhe) node[below,fill=white]{PEC};
\draw[very thick] (0.5+\breite,0) --+(\breite,0);
\filldraw[pattern=crosshatch,draw=white] (1+2*\breite,-\höhe) rectangle +(\breite,\höhe) node[below,fill=white]{PMC};
\draw[very thick] (1+2*\breite,0) --+(\breite,0);
\filldraw[pattern=crosshatch,draw=white] (1.5+3*\breite,-\höhe) rectangle +(\breite,\höhe) node[below,fill=white]{PMC};
\draw[very thick] (1.5+3*\breite,0) --+(\breite,0);

\draw[->,color=red,thick] (0.2,0.2)--+(0,0.4) node[left] {$\vec{J}$};
\draw[->,color=red,thick] (0.5,0.2)--+(0.4,0);

\draw[->>,color=green,thick] (0.5+\breite,0) ++(0.2,0.2)--+(0,0.4) node[left] {$\vec{M}$};
\draw[->>,color=green,thick] (0.5+\breite,0) ++(0.5,0.2)--+(0.4,0);

\draw[->,color=red,thick] (1+2*\breite,0) ++(0.2,0.2)--+(0,0.4) node[left] {$\vec{J}$};
\draw[->,color=red,thick] (1+2*\breite,0) ++(0.5,0.2)--+(0.4,0);

\draw[->>,color=green,thick] (1.5+3*\breite,0) ++(0.2,0.2)--+(0,0.4) node[left] {$\vec{M}$};
\draw[->>,color=green,thick] (1.5+3*\breite,0) ++(0.5,0.2)--+(0.4,0);


\draw[dashed,very thick] (0,\abstand) --+(\breite,0);
\draw[dashed,very thick] (0.5+\breite,\abstand) --+(\breite,0);
\draw[dashed,very thick] (1+2*\breite,\abstand) --+(\breite,0);
\draw[dashed,very thick] (1.5+3*\breite,\abstand) --+(\breite,0);

\draw[->,color=red,thick] (0.2,0.2+\abstand)--+(0,0.4);
\draw[->,color=red,thick] (0.5,0.2+\abstand)--+(0.4,0);
\draw[<-,color=red,thick] (0.2,-0.2+\abstand)--+(0,-0.4);
\draw[<-,color=red,thick] (0.5,-0.2+\abstand)--+(0.4,0);

\draw[->>,color=green,thick] (0.5+\breite,+\abstand) ++(0.2,0.2)--+(0,0.4);
\draw[->>,color=green,thick] (0.5+\breite,+\abstand) ++(0.5,0.2)--+(0.4,0);
\draw[->>,color=green,thick] (0.5+\breite,+\abstand) ++(0.2,-0.2)--+(0,-0.4);
\draw[->>,color=green,thick] (0.5+\breite,+\abstand) ++(0.5,-0.2)--+(0.4,0);

\draw[->,color=red,thick] (1+2*\breite,+\abstand) ++(0.2,0.2)--+(0,0.4);
\draw[->,color=red,thick] (1+2*\breite,+\abstand) ++(0.5,0.2)--+(0.4,0);
\draw[->,color=red,thick] (1+2*\breite,+\abstand) ++(0.2,-0.2)--+(0,-0.4);
\draw[->,color=red,thick] (1+2*\breite,+\abstand) ++(0.5,-0.2)--+(0.4,0);

\draw[->>,color=green,thick] (1.5+3*\breite,+\abstand) ++(0.2,0.2)--+(0,0.4);
\draw[->>,color=green,thick] (1.5+3*\breite,+\abstand) ++(0.5,0.2)--+(0.4,0);
\draw[<<-,color=green,thick] (1.5+3*\breite,+\abstand) ++(0.2,-0.2)--+(0,-0.4);
\draw[<<-,color=green,thick] (1.5+3*\breite,+\abstand) ++(0.5,-0.2)--+(0.4,0);


%\filldraw[pattern=dots,draw=white] (0,0) circle (3.3);
%\filldraw[color=white,draw=black] (0,0) circle (3);
%\filldraw[pattern=dots,draw=black] (1,1) circle (1);
%\draw node[fill=white] at (1,1) {$V_a$};
%\draw node at (-0.5,-1.5) {$V_b$};
%
%\draw[->] (-3,0)--(-2.25,0) node[below] {$\vec{n}$};
%\draw[->] (1,0)--(1,-0.75) node[right] {$\vec{n}$};
%\draw (-1,2.75) -- (0,2.25) node[fill=white] {$A$} -- (0.75,2.1);
%
%\draw[->] (-2,1)--+(0.5,0.75);
%\draw[->] (-1.8,0.8)--+(0.5,0.75);
%\draw[->] (-1.6,0.6)--+(0.5,0.75);
%\draw (-2.0,0.7) node[fill=white] {$\vec{J}(\vec{r})$};

%\filldraw [color=black!10] (0,-0.5) rectangle(3,1);
%\coordinate (center) at (1.5,1);
%\draw (center) -- ++(30:1.3) ;
%\draw [->] (center) -- ++(150:1.3);
%\draw [dashed] (center) -- ++(0,1);
%\draw [dashed] (center) -- ++(0,-1.5);
%
%\draw (center) +(0,0.75)arc [radius=0.75,start angle=90, end angle=30];
%\draw (center) +(0,0.75)arc [radius=0.75,start angle=90, end angle=150];
%
%\path (center) ++(75:1) node{$\theta_1$};
%\path (center) ++(105:1) node{$\theta_1$};
%
%\draw [->] (center) -- ++(250:1.3);
%\draw (center) +(0,-0.75)arc [radius=0.75,start angle=270, end angle=250];
%\path (center) ++(260:1) node{$\theta_2$};
%
%\path (center) ++(15:1.3) node{$\epsilon_{r1}$};
%\path (center) ++(345:1.3) node{$\epsilon_{r2}$};
%\path (center) ++(0:1.2) node{$\wedge$};
\end{tikzpicture}
\end{center}
\end{description}