\subsection*{Allgemeine Formeln}
Viele Größen können komplex werden und werden daher nicht gesondert markiert, $U$ ist z.B. die komplexe Spannungsamplitude.

\begin{description}
\item[Impedanz]
\begin{equation*}
Z = R + jX
\end{equation*}

\item[Admittanz]
\begin{equation*}
Y = G + jB
\end{equation*}

\item[Impedanz bestimmt Verhältnis von $U,I$]
\begin{equation*}
Z = \frac{U}{I}
\end{equation*}

\item[Scheinleistung]
\begin{equation*}
|P_S| = \left\vert P_W + j P_B \right\vert = \sqrt{P_W^2 + P_B^2} = \frac{1}{2} |U||I|\\
\end{equation*}

\item[Leistungsanpassung] Erlaubt maximale Leistungsübertragung
\begin{equation*}
Z = Z^*_i
\end{equation*}

\item[Abgegebene Generatorleistung]
\begin{equation*}
P_G = |a_G|^2 \frac{1-|\Gamma_1|^2}{|1-\Gamma_E\Gamma_1|^2}
\end{equation*}

\item[Maximale verfügbare Generatorleistung] Innenwiderstand $R_i = \Re\left\{Z_i\right\}$
\begin{equation*}
P_\text{max} = \frac{|U_0|^2}{8R_i} = \frac{|a_G|^2}{1-|\Gamma_1|^2}
\end{equation*}

\item[Prinzip der durchgehenden Wirkleistung] Gültig bei verlustlosen Zweitoren. Wenn ein Zweitor verlustbehaftet ist, wird es durch das Vorziehen der verlustbehafteten Komponenten wieder verlustfrei.
\begin{align*}
\left\vert \frac{U_n}{U_1} \right\vert = \sqrt{\frac{G_1}{G_n}} \\
\left\vert \frac{I_n}{I_1} \right\vert = \sqrt{\frac{R_1}{R_n}} \\
\end{align*}
Da $R \neq \frac{1}{G}$ erfolgt die Umrechnung zwischen $R$ und $G$ durch die folgenden Näherungsformeln:

\begin{tabular}{ccc}
$G \gg |B|$ & $R \approx \frac{1}{G}$ & $X \approx - \frac{B}{G^2}$ \\
$|B| \gg G$ & $R \approx \frac{G}{B^2}$ & $X \approx - \frac{1}{B}$ \\
$R \gg |X|$ & $G \approx \frac{1}{R}$ & $B \approx - \frac{X}{R^2}$ \\
$|X| \gg R$ & $G \approx \frac{R}{X^2}$ & $B \approx - \frac{1}{X}$ \\
\end{tabular}
\end{description}