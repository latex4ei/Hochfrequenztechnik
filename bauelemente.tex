\begin{center}
\begin{tabular}{ccc} \toprule
Bezeichnung & Symbol & Einheit \\ \midrule
Güte des Elements & $Q$ & \\
Verlustwinkel & $\delta$ & \\
Scheinbare Induktivität & $L^*$ & \si{\henry} \\
Scheinbare Kapazität & $C^*$ & \si{\farad} \\
\bottomrule
\end{tabular}
\end{center}

\begin{description}
\item[Widerstand] Vereinfachend für kleine $R$ nur $L_s$ und für große $R$ nur $C_p$.
\begin{center}
\begin{circuitikz}[scale=0.5, every node/.style={scale=1}]
\node (C1) at (0,1.3) {};
\node (C2) at (5,1.3) {};
\node (L1) at (0,0) {};
\node (L2) at (2.5,0) {};
\node (R2) at (5,0) {};

\draw (C1) to [capacitor=$C_p$] (C2);
\draw (L1) to [american inductor=$L_s$] (L2) to [R=$R$] (R2);

\draw (C1) to [short, -*] (L1);
\draw (C2) to [short, -*] (R2);

\draw (L1) to [short, -o] ++(-0.75,0);
\draw (R2) to [short, -o] ++(0.75,0);
\end{circuitikz}
\end{center}
\begin{align*}
Z_\text{ideal} &= j \omega L \\
Z_\text{real} &= \frac{j\omega L}{1 - \omega^2 LC_p} \\
Q_L &= \frac{\omega L}{R_s} = \frac{1}{\tan \delta_L} \\
R_s &= R_\text{Fe} + R_\text{Cu} = \omega L \tan\delta_\mu + R_\text{Cu} \\
L^* &= \frac{L}{1-\omega^2 C_p L} > L
\end{align*}

\item[Induktivität] Wird $\mu_r$ erhöht, so steigt $L$. Überhalb der Resonanzfrequenz verhält sich die reale Induktivität kapazitiv.
\begin{center}
\begin{circuitikz}[scale=0.5, every node/.style={scale=1}]
\node (C1) at (0,1.3) {};
\node (C2) at (5,1.3) {};
\node (L1) at (0,0) {};
\node (L2) at (2.5,0) {};
\node (R2) at (5,0) {};

\draw (C1) to [capacitor=$C_p$] (C2);
\draw (L1) to [american inductor=$L$] (L2) to [R=$R_s$] (R2);

\draw (C1) to [short, -*] (L1);
\draw (C2) to [short, -*] (R2);

\draw (L1) to [short, -o] ++(-0.75,0);
\draw (R2) to [short, -o] ++(0.75,0);
\end{circuitikz}
\end{center}
\begin{align*}
Z_\text{ideal} &= j \omega L \\
Z_\text{real} &= \frac{j\omega L}{1 - \omega^2 LC_p} \\
Q_L &= \frac{\omega L}{R_s} = \frac{1}{\omega L G_p} = \frac{1}{\tan \delta_L} \\
R_s &= R_\text{Fe} + R_\text{Cu} = \omega L \tan\delta_\mu + R_\text{Cu} \\
L^* &= \frac{L}{1-\omega^2 C_p L} > L
\end{align*}

\item[Induktivität durch kurzgeschlossene Leitung]
\begin{align*}
Z &= jX = j Z_L \tan( \beta l)
\intertext{falls $l \ll \lambda_0$}
X &= \omega L' l \\
\end{align*}

\item[Kapazität] Überhalb der Resonanzfrequenz verhält sich die reale Kapazität induktiv.
\begin{center}
\begin{circuitikz}[scale=0.5, every node/.style={scale=1}]
%\node [contact] (contact 1) at (0,2) {};
%\node [contact] (contact 2) at (6,2) {};
%\node [contact] (contact 3) at (3,2) {};
%
%\draw (contact 1) to [inductor={info'=$L_s$, midway}] (contact 3) to [capacitor={info'=$C$,midway}] (contact 2);
%\draw (contact 3) --++(up:1) to [resistor={info=$G_p$}] ++(right:3) to (contact 2);

%
\node (C1) at (2.5,1.5) {};
\node (C2) at (5,1.5) {};
\node (L1) at (0,0) {};
\node (L2) at (2.5,0) {};
\node (R2) at (5,0) {};

\draw (C1) to [R=$G_p$] (C2);
\draw (L1) to [american inductor=$L_s$] (L2) to [C=$C$] (R2);

\draw (C1) to [short, -*] (L2);
\draw (C2) to [short, -*] (R2);

\draw (L1) to [short, -o] ++(-0.75,0);
\draw (R2) to [short, -o] ++(0.75,0);
\end{circuitikz}
\end{center}
\begin{align*}
Y_\text{ideal} &= j \omega C \\
Y_\text{real} &= \frac{j \omega C}{1 - \omega^2 L_sC} \\
Q_C &= \frac{\omega C}{G_p} = \frac{1}{\omega C R_s} = \frac{1}{\tan \delta_C} \\
\tan \delta_C &= \frac{\epsilon_r''}{\epsilon_r'} \text{ für homogenes Dielektrikum}\\
C^* &= \frac{C}{1 - \omega^2 L_s C} > C \\
\end{align*}

\item[Kapazität durch leerlaufende Leitung]
\begin{align*}
Y &= jB = j \frac{1}{Z_L} \tan( \beta l)
\intertext{falls $l \ll \lambda_0$}
B &= \omega C' l \\
\end{align*}

% HFVO Folie 12
\item[Schottky-Diode] Metall-Halbleiter-Übergang, fehlende Minoritätsträger ermöglichen Nutzung für höchste Frequenzen. Anwedungen: Gleichrichter, Detektoren, Mischer.

% HFVO Folie 14
\item[Kapazitätsdiode] Betrieb in Sperrrichtung. Hohe Sperrspannung bedeutet niedrige Kapazität und umgekehrt. Anwendungen: Abstimmung von Schwingkreisen, Filtern.

% HFVO Folie 15
\item[Tunneldiode] Kennlinie besitzt wegen Tunneleffekt Abschnitt mit negativem Widerstand. Kurzschlussstabil. Anwendung: Breitbandige Verstärkung von Signalen.

% HFVO Folie 16
\item[Impatt-Diode] Kennlinie besitzt durch Laufzeitverzerrungen Abschnitt mit negativem Widerstand. Anwendung: Verstärkung von Signalen.

% HFVO Folie 20
\item[PIN-Diode] Erzeugt mittels eines Gleichstroms einen Wechselstromwiderstand. Anwendungen: HF-Schalter, Wellenwiderstandsanpassung.

% HFVO Folie 23
\item[Gunn-Element] Durch Gunn-Effekt entsteht ein Kennlinienabschnitt mit negativem Widerstand. Anwendung: Oszillator.

\end{description}