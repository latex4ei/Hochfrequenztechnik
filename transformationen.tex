\begin{description}
\item[Polarkoordinaten $\longmapsto$ Kartesisch]
\begin{equation*}
Z = \Re + j\Im \qquad \Re = |r| \cos \varphi \qquad \Im = |r| \sin \varphi
\end{equation*}

\item[Kartesisch $\longmapsto$ Polarkoordinaten]
\begin{equation*}
Z = r e^{j \varphi} \qquad r = \sqrt{\text{Re}^2 + \text{Im}^2} \qquad \varphi = \left\{ \begin{array}{cc}
+\arccos \frac{\Re}{r} & \Im \geq 0 \\ 
-\arccos \frac{\Re}{r} & \Im < 0
\end{array} \right.
\end{equation*}

\item[Dezibel]
\begin{equation*}
\begin{array}{cc}
\SI{1}{\neper} = \SI{8,686}{\dB} & \SI{0}{\dBd} = \SI{2,15}{\dBi} \\ 
\SI{1}{\watt} = \SI{30}{\dBm} = \SI{0}{\dB} & \SI{1}{\milli\watt} = \SI{0}{\dBm} = \SI{-30}{\dB} \\
 & \\
P_{\si{\dB}} = 10\log_{10}\left( P_{\si{\watt}} \right) & P_{\si{\watt}} = 10^{\left( \frac{P_{\si{\dB}}}{10} \right)} \\
U_{\si{\dB}} = 20\log_{10}\left( U_{\si{\volt}} \right) & U_{\si{\volt}} = 10^{\left( \frac{U_{\si{\dB}}}{20} \right)} \\
U_{\si{dBuV}} = 107 + P_{\si{\dBm}} & \text{gilt nur für \SI{50}{\ohm}}
\end{array}
\end{equation*}

\item[Dyaden und Vektoren (heuristisch)]
\begin{align*}
\overleftrightarrow{D} &= \left[
\begin{array}{ccc}
\vec{e}_x\vec{e}_x & \vec{e}_x\vec{e}_y & \vec{e}_x\vec{e}_z \\ 
\vec{e}_y\vec{e}_x & \vec{e}_y\vec{e}_y & \vec{e}_y\vec{e}_z \\ 
\vec{e}_z\vec{e}_x & \vec{e}_z\vec{e}_y & \vec{e}_z\vec{e}_z \\ 
\end{array} \right] \qquad
\vec{a} =  \left[
\begin{array}{c}
\vec{e}_x \\ 
\vec{e}_y \\ 
\vec{e}_z \\ 
\end{array} \right]\\
\intertext{Als Beispiel wird folgende Dyade nebst Vektor betrachtet:}
\overleftrightarrow{D}_1 &= 
\vec{e}_x\vec{e}_x + \vec{e}_x\vec{e}_y + \vec{e}_y\vec{e}_z
=
\left[
\begin{array}{ccc}
1 & 1 & 0 \\ 
0 & 0 & 1 \\ 
0 & 0 & 0 \\ 
\end{array} \right],
\vec{a}_1 = 
\vec{e}_x - 2\vec{e}_y
= \left[
\begin{array}{c}
1 \\ 
-2 \\ 
0 \\ 
\end{array}\right]\\
\vec{a}_1 \times \overleftrightarrow{D}_1 &= \left[
\begin{array}{c}
1 \\ 
-2 \\ 
0 \\ 
\end{array}\right] \times \left[
\begin{array}{ccc}
1 & 1 & 0 \\ 
0 & 0 & 1 \\ 
0 & 0 & 0 \\ 
\end{array} \right] =
\left[
\begin{array}{ccc}
0 & 0 & 0 \\ 
0 & 0 & 0 \\ 
2 & 2 & 1 \\ 
\end{array} \right]
\intertext{Nun wandelt man noch die Matrix in die ursprüngliche Schreibweise um:}
\vec{a}_1 \times \overleftrightarrow{D}_1 &= 2\vec{e}_z\vec{e}_x + \vec{e}_z\vec{e}_y + \vec{e}_z\vec{e}_z
\end{align*}
Das Skalarprodukt $\overleftrightarrow{D} \cdot \vec{a}$ entspricht einer normalen Matrix-Vektor-Multiplikation nach dem bekannten Schema \textit{Zeile mal Spalte} und ist deswegen nicht explizit aufgeführt. Man erhält einen $3\times 1$-Vektor. Das Kreuzprodukt $\vec{a} \times \overleftrightarrow{D}$ entspricht einer normalen Kreuzproduktbildung, jedoch wird jede Spalte von $\overleftrightarrow{D}$ einzeln mit $\vec{a}$ kreuzweise multipliziert. Man erhält wieder eine $3\times 3$-Matrix.
\end{description}
