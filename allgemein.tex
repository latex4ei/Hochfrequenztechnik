Viele Größen können komplex werden und werden daher nicht gesondert markiert, $U$ ist z.B. die komplexe Spannungsamplitude.
\begin{description}
\item[Impedanz]
\begin{equation*}
Z = R + jX
\end{equation*}

\item[Admittanz]
\begin{equation*}
Y = G + jB
\end{equation*}

\item[Impedanz bestimmt Verhältnis von $U,I$]
\begin{equation*}
Z = \frac{U}{I}
\end{equation*}

\item[Scheinleistung]
\begin{equation*}
|P_S| = \left\vert P_W + j P_B \right\vert = \sqrt{P_W^2 + P_B^2} = \frac{1}{2} |U||I|\\
\end{equation*}

\item[Leistungsanpassung] Erlaubt maximale Leistungsübertragung
\begin{equation*}
Z = Z^*_i
\end{equation*}
\end{description}