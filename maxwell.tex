\subsection*{Maxwellsche Gleichungen}
\begin{center}
\begin{tabular}{ccc} \toprule
Bezeichnung & Symbol & Einheit \\ \midrule
Elektrische Feldstärke & $\vec{e}$ & \si{\volt\per\meter}\\
Elektrische Verschiebungsflussdichte & $\vec{D}_m$ & \si{\ampere\second\per\meter^2}\\
Magnetische Feldstärke & $\vec{h}$ & \si{\ampere\per\meter}\\
Magnetische Flussdichte & $\vec{b}$ & \si{\volt\second\per\meter^2}\\
Elektrische Raumladungsdichte & $\vec{\rho}$ & \si{\ampere\second\per\meter^3}\\
Elektrische Stromdichte & $\vec{j}$ & \si{\ampere\per\meter^2}\\
Wellenzahl & $k$ & \si{\per\meter}\\
Oberflächenimpedanz & $Z_S$ & \si{\ohm} \\
Oberflächenresistanz & $R_S$ & \si{\ohm}\\
\bottomrule
\end{tabular}
\end{center}


\begin{description}
\item[Durchflutungsgesetz (Ampère)]
\begin{align*}
\oint\limits_c \vec{h} \cdot d\vec{s} = \iint\limits_{A(V)} \left( \vec{j} + \frac{\partial \vec{D}_m}{\partial t}\right ) \cdot d\vec{A} \\
\rot{\vec{h}} = \vec{j} + \d{\vec{D}_m}{t} \\
\rot{\vec{H}} = \vec{J} + j\omega \vec{D} = \kappa \vec{E} + j\omega \epsilon_0 \epsilon_r \vec{E} \\
\end{align*}

\item[Induktionssatz (Faraday)]
\begin{align*}
\oint\limits_c \vec{e} \cdot d\vec{s} = - \iint\limits_{A(V)} \frac{\partial \vec{b}}{\partial t} \cdot d\vec{A} \\
\rot{\vec{e}} = - \d{\vec{b}}{t} \\
\rot{\vec{E}} = - j\omega\vec{B} = - j \omega \mu_0 \mu_r\vec{H}\\
\end{align*}

\item[Gaußsches Gesetz]
\begin{align*}
\oint\limits_S \vec{D}_m \cdot d\vec{A} = \iiint\limits_V \rho \cdot dV \\
\div{\vec{D}_m} = \rho \\
\div{\vec{D}} = \div{\epsilon_0 \epsilon_r \vec{E}} = \rho \\
\end{align*}

\item[Gaußsches Gesetz für die magnetische Flussdichte]
\begin{align*}
\oint\limits_S \vec{b} \cdot d\vec{A} = \int\limits_V \rho_m \cdot dV \\
\div{\vec{b}} = 0 \\
\div{\vec{B}} = 0\\
\end{align*}

\item[Materialgesetze]
\begin{align*}
\vec{D}_m = \epsilon\vec{e} = \epsilon_0 \epsilon_r \vec{e} \\
\vec{b} = \mu \vec{h} = \mu_0 \mu_r \vec{h} \\
\vec{j} = \kappa \vec{e} \\
\vec{D} = \epsilon\vec{E} = \epsilon_0 \epsilon_r \vec{E} \\
\vec{B} = \mu \vec{H} = \mu_0 \mu_r \vec{H} \\
\vec{J} = \kappa \vec{E}
\end{align*}

\item[Randbedingungen für Tangentialkomponenten]
\begin{align*}
\text{Normalfall: }&\vec{n}\times\left(\vec{h}_1 - \vec{h}_2 \right) = 0 \\
\text{Oberflächenströme: }&\vec{n}\times\left(\vec{h}_1 - \vec{h}_2 \right) = \vec{j}_F \\
\text{Normalfall: }&\vec{n}\times\left(\vec{e}_1 - \vec{e}_2 \right) = 0 \\
\end{align*}

\item[Randbedingungen für Normalkomponenten]
\begin{align*}
\text{Normalfall: }&\vec{n}\cdot\left(\vec{D}_{m1} - \vec{D}_{m2} \right) = 0 \\
\text{Oberflächenladungen: }&\vec{n}\cdot\left(\vec{D}_{m1} - \vec{D}_{m2} \right) = \rho_F \\
\text{Normalfall: }&\vec{n}\cdot\left(\vec{b}_1 - \vec{b}_2 \right) = 0 \\
\end{align*}

\item[Kontinuitätsgleichung des Stroms]
\begin{align*}
\oint\limits_{A(V)} \vec{j} \cdot d\vec{A} = - \int\limits_V \frac{\partial \rho}{\partial t} \cdot dV \\
\div{\vec{j}} = - \frac{\partial \rho}{\partial t} \\
\div{\vec{J}} = - j \omega \rho \\
\end{align*}

\item[Elektromagnetische Bilanzgleichung]
\begin{align*}
\oiint\limits_V \vec{S} \cdot d\vec{A} &= -P_v + j\omega 2(\bar{w}_e + \bar{w}_m) \\
\div{\vec{J_\text{em}}} &= \Pi_\text{em} - \frac{\partial w_\text{em}}{\partial t} \\
\div{\frac{1}{2}\vec{E}\times\vec{H^*}} &= -\vec{j}\cdot\vec{E} - \frac{\partial(w_e + w_m)}{\partial t}
\end{align*}

\item[Wellengleichung]
\begin{equation*}
\rot\rot\vec{E} - k^2\vec{E} = -j \omega \mu_0 \mu_r \vec{J}
\end{equation*}

\item[Helmholtz-Gleichung]
\begin{equation*}
\divgrad{\vec{E}} + k^2 \vec{E} = 0
\end{equation*}

\item[Skineffekt] Bei Eindringtiefe der Welle von $1\delta$: Feldstärke $E_0$ fällt auf $0,37 E_0$ oder $\frac{1}{e}$ ab.
\begin{align*}
\delta &= \frac{1}{k} = \sqrt{\frac{2}{\omega \mu \kappa}} \\
Z_S &= R_S(1+j) = (1+j) \sqrt{\frac{\omega \mu_0 \mu_r}{2 \kappa}}
\end{align*}
\end{description}