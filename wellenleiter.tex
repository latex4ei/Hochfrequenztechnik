\begin{description}
\item[Allgemeine Formeln] \strut
\begin{center}
\begin{tabular}{ccc} \toprule
Bezeichnung & Symbol & Einheit \\ \midrule
Leitungswellenwiderstand & $Z_L$ &  \si{\ohm}\\
Dämpfungskonstante & $\alpha$ & \si{\neper\per\meter} \\
Phasenkonstante & $\beta$ & \si{\rad\per\meter} \\
Phasengeschwindigkeit & $v_\text{ph}$ & \si{\meter\per\second}\\
Reflexionsfaktor & $\Gamma$ &  \\
Voltage Standing Wave Ratio & VSWR &  \\
Induktivitätsbelag & $L'$ & \si{\henry\per\meter} \\
Kapazitätsbelag & $C'$ & \si{\farad\per\meter} \\
Widerstandsbelag & $R'$ & \si{\ohm\per\meter} \\
Leitwertsbelag & $G'$ & \si{\siemens\per\meter} \\
Leitfähigkeit & $\kappa$ &  \si{\siemens\per\meter}\\
\bottomrule
\end{tabular}
\end{center}
\begin{align*}
	U(z) &= U_h e^{-\gamma z} + U_r e^{+\gamma z} \\
	I(z) &= U_h e^{-\gamma z} + I_r e^{+\gamma z} \\
	Z_L &= \frac{U_h}{I_h} = \frac{U_r}{I_r} = \sqrt{\frac{R' + j \omega L'}{G' + j \omega C'}} \approx \sqrt{\frac{L'}{C'}} = \sqrt{\frac{\mu_0 \mu_r}{\epsilon_0 \epsilon_r}} \\
	\gamma &= \alpha + j \beta = \sqrt{(R' + j \omega L')(G' + j \omega C')} \\
	\alpha &= \frac{R'}{2Z_L} \\
	\beta &= \frac{2\pi}{\lambda} = \omega \sqrt{L' C'} = \omega \sqrt{\epsilon_0 \epsilon_r \mu_0 \mu_r} \\
	v_{\text{ph}} &= \frac{\omega}{\beta} = \frac{1}{\sqrt{L' C'}} = \frac{c_o}{\sqrt{\mu_r \epsilon_r}} \text{ mit } c_0 = \frac{1}{\sqrt{\epsilon_0 \mu_0}} \\
	\Gamma &= \frac{b_i}{a_i} = \frac{U_r}{U_h} = \frac{I_r}{I_h} = \frac{Z_0 - Z_L}{Z_0 + Z_L} \\
	\text{VSWR} &= \frac{1+|\Gamma|}{1-|\Gamma|} = \frac{U_\text{max}}{U_\text{min}} \\
	Z_{\lambda / 4} &= \sqrt{Z_E Z_\text{Last}}\\
\end{align*}

\item[Koaxialleitung] Grundmode ist TEM, es herrscht Dispersionsfreiheit
\begin{center}
\begin{tabular}{ccc} \toprule
Bezeichnung & Symbol & Einheit \\ \midrule
Außenleiter Innendurchmesser & $D$ &  \si{\meter}\\
Innenleiter Durchmesser & $d$ &  \si{\meter}\\
\bottomrule
\end{tabular}
\end{center}
\begin{center}
\begin{tikzpicture}[scale=0.2, every node/.style={scale=0.8}]
\filldraw[pattern=north west lines] (0,0) circle (0.5);
\filldraw[even odd rule,pattern=north west lines] (0,0) circle (2) (0,0) circle (2.5);
\draw[|-|] (-2.8,-2) -- (-2.8,2) node[anchor=east,midway]{D};
\draw[|-|] (0.7,-0.5) -- (0.7,0.5) node[anchor=west,midway]{d};
\end{tikzpicture}
\end{center}
\begin{align*}
Z_L &\approx \sqrt{\frac{L'}{C'}} = Z_{F0} \frac{\ln(\frac{D}{d})}{2\pi\sqrt{\epsilon_r}} \\
L' &= \frac{\mu_0 \mu_r}{2\pi} \ln\left(\frac{D}{d}\right) \\
C' &= \frac{2\pi \epsilon_0 \epsilon_r}{\ln \left(\frac{D}{d}\right)} \\ 
R' &= \sqrt{\frac{\omega \mu}{2 \kappa}} \cdot \frac{1}{\pi}\left( \frac{1}{D} + \frac{1}{d} \right) \\
G' &= - \frac{\kappa}{\epsilon_0 \epsilon_r} C' \\
\end{align*}
\begin{align*}
\alpha = \alpha_c + \alpha_d =& \sqrt{\frac{f \mu}{\sigma \pi}} \left( \frac{2}{d} + \frac{2}{D} \right) \frac{1}{4Z_L} + \\
&+ \pi f \sqrt{\epsilon_r} \frac{tan \delta}{c_0}
\end{align*}

\item[Zweidrahtleitung/Doppelleitung] Grundmode ist TEM
\begin{center}
\begin{tabular}{ccc} \toprule
Bezeichnung & Symbol & Einheit \\ \midrule
Leiterabstand & $d$ &  \si{\meter}\\
Leiterradius & $a$ &  \si{\meter}\\
\bottomrule
\end{tabular}
\end{center}
\begin{center}
\begin{tikzpicture}[scale=0.2, every node/.style={scale=0.8}]
\filldraw [pattern=north west lines] (-5,0) circle(1);
\filldraw [pattern=north west lines] (5,0) circle(1);
\node [label=left:2a] at (-6,0) {};
\draw [|-|] (-4,-1.4) -- (4,-1.4);
\draw [|-|] (-6.4,-1) -- (-6.4,1);
\node [label=below:d] at (0,-1.4) {};
\end{tikzpicture}
\end{center}
\begin{align*}
Z_L &= \frac{Z_{F0}}{\pi\sqrt{\epsilon_r}} \ln\left( \frac{d}{a} \right) \\
L' &= \frac{\mu_0 \mu_r}{\pi} \ln\left(\frac{d}{a}\right) \\
C' &= \frac{\pi \epsilon_0 \epsilon_r}{\ln\left(\frac{d}{a}\right)} \\
R' &= \sqrt{\frac{\omega \mu_0 \mu_r}{2 \kappa}} \cdot \frac{1}{\pi a} \\
G' &= \frac{\kappa}{\epsilon_0 \epsilon_r} C' \\
\end{align*}

\item[Streifenleitung/Bandleitung] Grundmode ist TEM
\begin{center}
\begin{tabular}{ccc} \toprule
Bezeichnung & Symbol & Einheit \\ \midrule
Leiterabstand & $a$ &  \si{\meter}\\
Leiterbreite & $b$ &  \si{\meter}\\
\bottomrule
\end{tabular}
\end{center}
\begin{center}
\begin{tikzpicture}[scale=0.2, every node/.style={scale=0.8}]
\filldraw [pattern=north west lines] (0,0) rectangle(12,1);
\filldraw [pattern=north west lines] (0,-3) rectangle(12,-2);
\node [label=below:b] at (6,-3.5) {};
\node [label=right:a] at (12.9,-1) {};
\draw [|-|] (0,-3.9) -- (12,-3.9);
\draw [|-|] (12.9,-2) -- (12.9,0);
\end{tikzpicture}
\end{center}
\begin{align*}
Z_L &= \frac{Z_{F0}}{\pi\sqrt{\epsilon_r}} \frac{b}{a}\\
L' &= \mu_0 \mu_r \frac{a}{b}\\
C' &= \epsilon_0 \epsilon_r \frac{b}{a}\\
R' &= \sqrt{\frac{\omega \mu_0 \mu_r}{2 \kappa}} \cdot \frac{2}{b}\\
G' &= - \frac{\kappa}{\epsilon_0 \epsilon_r} C' \\
\end{align*}

\item[Mikrostreifenleitung] Grundmode ist Quasi-TEM
\begin{center}
\begin{tabular}{ccc} \toprule
Bezeichnung & Symbol & Einheit \\ \midrule
Substrathöhe & $h$ & \si{\meter} \\
Leiterbreite & $w$ & \si{\meter} \\
Effektive Leiterbreite & $w_\text{eff}$ &  \si{\meter}\\
Effektive Dielektrizitätszahl & $\epsilon_{r,\text{eff}}$ \\
Cutoff-Frequenz für 1. Mode & $f_{c,\text{HE1}}$ & \si{\hertz} \\
Güte der Leitung & $Q$ & \\
Substratwellenlänge & $\lambda_\epsilon$ & \si{\meter} \\
\bottomrule
\end{tabular}
\end{center}

Eigenschaften:
\begin{itemize}
\item[+] Verlustarm
\item[+] Serienelemente leicht integrierbar
\item[+] Einfache Fertigung
\item[-] Parallelelemente schwierig
\item[-] Leiterbreite nicht skalierbar
\end{itemize}

\begin{center}
\begin{tikzpicture}[scale=0.2, every node/.style={scale=0.8}]
\filldraw [pattern=north west lines] (4,0) rectangle(8,1);
\filldraw [pattern=north west lines] (0,-3) rectangle(12,-2);
\draw (0,-2) rectangle(12,0);
\node [label=above:$w$] at (6,1.9) {};
\node [label=left:$h$] at (-0.9,-1) {};
\draw [|-|] (4,1.9) -- (8,1.9);
\draw [|-|] (-0.9,-2) -- (-0.9,0);
\node at (6,-1) {$\epsilon_r$};

\draw [->] (13,-1) -- ++(4,0);

\filldraw [pattern=north west lines] (18,0) rectangle(30,1);
\filldraw [pattern=north west lines] (18,-3) rectangle(30,-2);
\draw (18,-2) rectangle(30,0);
\node [label=above:$w_\text{eff}$] at (24,1.9) {};
\node [label=right:$h$] at (30.9,-1) {};
\draw [|-|] (18,1.9) -- (30,1.9);
\draw [|-|] (30.9,-2) -- (30.9,0);
\node at (24,-1) {$\epsilon_{r,\text{eff}}$};
\end{tikzpicture}
\end{center}
Kapazitäten und Induktivitäten können durch einfache Leiterbreitenvariation erzeugt werden.
\begin{center}
\begin{tikzpicture}[scale=0.2, every node/.style={scale=1}]
\draw (-7,3) -- (7,3);
\draw (-7,-3) -- (7,-3);

\draw[pattern=north west lines] (-7,-0.5) -- ++(0,1) -- ++(3,0) -- ++(0,1.5) -- ++(1,0) -- ++(0,-1.5) -- ++(4,0) -- ++(0,-0.25) -- ++(3,0) -- ++(0,0.25) -- ++(3,0) -- ++(0,-1) -- ++(-3,0) -- ++(0,+0.25) -- ++(-3,0) -- ++(0,-0.25) -- ++(-4,0) --++(0,-1.5)-- ++(-1,0) -- ++(0,1.5) -- cycle;

\node at (-5.5,-2){$C_p$};
\node at (2.5,1.5){$L_s$};
\end{tikzpicture}
\end{center}
Hinweis: Näherungsformeln für Effektivwerte weichen je nach Vorlesung ab.
\begin{align*}
Z_L &= \frac{Z_{F0}}{\sqrt{\epsilon_{r,\text{eff}}}} \cdot \frac{h}{w_\text{eff}} \\
w_\text{eff} &= \frac{Z_{F0}}{Z_L} \cdot \frac{h}{\sqrt{\epsilon_{r,\text{eff}}}} \\
w_\text{eff} &= w + \frac{2h}{\pi} \cdot \left\{ 1 + \ln\left(2\pi ( \frac{w}{h} + 0,92) \right) \right\} \\
\sqrt{\epsilon_{r,\text{eff}}} &= \frac{\epsilon_r + 1}{2} + \frac{\epsilon_r - 1}{2} \cdot \frac{1}{\sqrt{1 + 10 \frac{h}{w}}} \\
f_{c,\text{HE1}} &= \frac{c_0}{2w_\text{eff}\sqrt{\epsilon_{r,\text{eff}}}} \\
Q &= \frac{\pi}{\lambda_\epsilon \alpha} \\
v_\text{ph} &= \frac{c_0}{\sqrt{\epsilon_{r,\text{eff}}}} \\
\beta &= \frac{2\pi}{\lambda_\epsilon} = \frac{2\pi}{\lambda_o} \sqrt{\epsilon_{r,\text{eff}}} = \frac{2\pi f}{c_0} \sqrt{\epsilon_{r,\text{eff}}} \\
\end{align*}

\item[Konforme Abbildung für Wellenwiderstandsberechnung] Für die folgenden Wellenleiter werden die Parameter über eine konforme Abbildung berechnet. %Viel Spaß.
\begin{align*}
\frac{K(k)}{K'(k)} &= \left\{ \begin{matrix}
\left[ \frac{1}{\pi} \ln \left( 2\frac{1+\sqrt{k'}}{1-\sqrt{k'}} \right) \right]^{-1} & 0 \leq k \leq \frac{1}{\sqrt{2}} \\ 
\frac{1}{\pi} \ln \left( 2\frac{1+\sqrt{k}}{1-\sqrt{k}} \right) & \frac{1}{\sqrt{2}} < k \leq 1
\end{matrix}  \right. \\
k' &= \sqrt{1-k^2}
\end{align*}

\item[Streifenleitung] Grundmode ist TEM
\begin{center}
\begin{tabular}{ccc} \toprule
Bezeichnung & Symbol & Einheit \\ \midrule
Substrathöhe & $h$ & \si{\meter} \\
Innenleiterbreite & $w$ & \si{\meter} \\
\bottomrule
\end{tabular}
\end{center}

\begin{center}
\begin{tikzpicture}[scale=0.2, every node/.style={scale=0.8}]
\filldraw [pattern=north west lines] (0,0) rectangle(12,1);
\filldraw [pattern=north west lines] (0,-6) rectangle(12,-5);
\filldraw [pattern=north west lines] (4,-2) rectangle(8,-3);
\draw (0,-5) rectangle(12,0);
\draw[|-|] (-1,-5) --+(0,5) node[midway,left] {$h$};
\draw[|-|] (4,-1.1) --+(4,0) node[midway,above] {$w$};
\node at (11,-1) {$\epsilon_r$};
\end{tikzpicture}
\end{center}

\begin{align*}
Z_L &= \frac{Z_{F0}}{4 \sqrt{\epsilon_r}} \frac{K'(k)}{K(k)} \\
k &= \tanh \left( \frac{\pi}{w} \frac{w}{h} \right)
\end{align*}

\item[Koplanarleitung] \strut
\begin{center}
\begin{tabular}{ccc} \toprule
Bezeichnung & Symbol & Einheit \\ \midrule
Substrathöhe & $h$ & \si{\meter} \\
Mittelleiterbreite & $w$ & \si{\meter} \\
Abstand & $s$ & \si{\meter} \\
Effektive Dielektrizitätszahl & $\epsilon_{r,\text{eff}}$ \\
\bottomrule
\end{tabular}
\end{center}
Eigenschaften:
\begin{itemize}
\item[+] Einseitige Metallisierung
\item[+] Serienelemente leicht integrierbar
\item[+] Parallelelemente leicht integrierbar
\item[-] Mittlere Verluste
\item[-] Even-odd-Moden (Unterdrückung von odd durch Drahtbrücken)
\end{itemize}
\begin{center}
\begin{tikzpicture}[scale=0.2, every node/.style={scale=0.8}]
\filldraw [pattern=north west lines] (0,0) rectangle(7,1);
\filldraw [pattern=north west lines] (10,0) rectangle(13,1);
\filldraw [pattern=north west lines] (16,0) rectangle(23,1);
\draw (0,-5) rectangle(23,0);
\draw[|-|] (-1,-5) --+(0,5) node[midway,left] {$h$};
\draw[|-|] (10,1.9) --+(3,0) node[midway,above] {$w$};
\draw[|-|] (7,1.9) --+(3,0) node[midway,above] {$s$};
\draw[|-|] (13,1.9) --+(3,0) node[midway,above] {$s$};
\node at (18,-3) {$\epsilon_{r,\text{eff}}$};
\end{tikzpicture}
\end{center}

\begin{align*}
Z_L &= \frac{Z_{F0}}{4 \sqrt{\epsilon_r}} \frac{K'(k)}{K(k)} \\
\epsilon_{r,\text{eff}} &= 1+\frac{\epsilon_r -1 }{2} \frac{K'(k)}{K(k)} \frac{K(k_1)}{K'(k_1)} \\
k &= \frac{w}{w+2s} \\
k_1 &= \frac{\sinh \frac{\pi w}{4h}}{\sinh \frac{\pi(w+2s)}{4h}}
\intertext{für $h\rightarrow \infty$}
k_1 &= k
\end{align*}

\item[Koplanare Streifenleitung] \strut
\begin{center}
\begin{tabular}{ccc} \toprule
Bezeichnung & Symbol & Einheit \\ \midrule
Leiterbreite & $w$ & \si{\meter} \\
Abstand zwischen den Leitungen & $s$ & \si{\meter} \\
\bottomrule
\end{tabular}
\end{center}
Verwendung:
\begin{itemize}
\item Speiseleitung für Antennen
\item Leitungstransformationen
\end{itemize}

\begin{center}
\begin{tikzpicture}[scale=0.2, every node/.style={scale=0.8}]
\filldraw [pattern=north west lines] (7,0) rectangle(10,1);
\filldraw [pattern=north west lines] (13,0) rectangle(16,1);
\draw (0,-5) rectangle(23,0);
\draw[|-|] (10,1.9) --+(3,0) node[midway,above] {$s$};
\draw[|-|] (7,1.9) --+(3,0) node[midway,above] {$w$};
\draw[|-|] (13,1.9) --+(3,0) node[midway,above] {$w$};
\node at (18,-3) {$\epsilon_{r,\text{eff}}$};
\end{tikzpicture}
\end{center}

\begin{align*}
Z_L &= \frac{Z_{F0}}{\sqrt{\epsilon_{r,\text{eff}}}} \frac{K(k)}{K'(k)} \\
k &= \frac{s}{s+2w} \\
\end{align*}

\item[Allgemeiner Hohlleiter] Grundmode ist TE oder TM
\begin{center}
\begin{tabular}{ccc} \toprule
Bezeichnung & Symbol & Einheit \\ \midrule
Phasengeschwindigkeit & $v_\text{ph}$ & \si{\meter\per\second} \\
Gruppengeschwindigkeit & $v_g$ & \si{\meter\per\second} \\
Wellenzahl & $k$ & \si{\per\meter} \\
\bottomrule
\end{tabular}
\end{center}
\begin{align*}
v_\text{ph} &= \frac{\omega}{\beta}
\intertext{ist häufig größer als $c_0$, während die Information/Energie mit}
v_g &= \left( \frac{d \beta}{d \omega} \right)^{-1} \leq c_0
\end{align*}
übertragen wird.

\item[Rechteckhohlleiter] \strut
\begin{center}
\begin{tabular}{ccc} \toprule
Bezeichnung & Symbol & Einheit \\ \midrule
Breite & $a$ & \si{\meter} \\
Höhe & $b$ & \si{\meter} \\
Eigenwert & $q_v$ & \si{\per\meter} \\
Ausbreitungsmaß & $\gamma_v$ & \si{\per\meter} \\
Cutoff-Frequenz & $f_{c,v}$ & \si{\hertz} \\
Cutoff-Wellenlänge & $\lambda_{c,v}$ & \si{\meter} \\
Transportierte Leistung & $P$ & \si{\watt} \\
\bottomrule
\end{tabular}
\end{center}
%\begin{center}
%\begin{tikzpicture}[scale=0.4]
%	\filldraw [pattern=north west lines, even odd rule] (0.25,0.25) rectangle(3.25,1.25) (0,0) rectangle(3.5,1.5);

\node at (1.75,-0.6){a};
\node at (-0.6,0.75){b};
%\end{tikzpicture}
%\end{center}

\begin{align*}
q_v &= \sqrt{\left( \frac{\pi m}{a} \right)^2 + \left( \frac{\pi n}{b} \right)^2 } \\
k^2 &= \omega^2 \epsilon_0 \epsilon_r \mu_0 \mu_r \\
\gamma_v &= \sqrt{q_v - k^2} \\
f_{c,v} &= \frac{c_0}{2\pi\sqrt{\epsilon_r \mu_r}} q_v \\
\lambda_{c,v} &= \frac{2\pi\sqrt{\epsilon_r \mu_r}}{q_v} =\frac{2\sqrt{\epsilon_r \mu_r}}{\sqrt{\left( \frac{m}{a} \right)^2 + \left( \frac{n}{b} \right)^2}} \\
P &= \frac{ab}{4} \frac{E_\text{max}^2}{Z_F} \\
Z_{F,E} &= \frac{\gamma_v}{j\omega \epsilon_0 \epsilon_r} = Z_{F0}\sqrt{1 - \left(\frac{f_c}{f}\right)^2 } = Z_{F0}\sqrt{1 - \left(\frac{\lambda_0}{\lambda_{c,v}}\right)^2 }\\
Z_{F,H} &= \frac{j\omega \mu_0 \mu_r}{\gamma_v} = \frac{Z_{F0}}{\sqrt{1 - \left(\frac{f_c}{f}\right)^2 }} = \frac{Z_{F0}}{\sqrt{1 - \left(\frac{\lambda_0}{\lambda_{c,v}}\right)^2 }} \\
\lambda_{H,mn} &= \lambda_{E,mn} = \frac{1}{\sqrt{\epsilon_r \mu_r}} \frac{\lambda_0}{\sqrt{1 - \left(\frac{\lambda_0}{\lambda_{c,v}}\right)^2 }} \\
v_\text{ph} &= \frac{c_0}{\sqrt{1 - \left(\frac{f_c}{f}\right)^2 }} = \frac{c_0}{\sqrt{1 - \left(\frac{\lambda_0}{\lambda_{c,v}}\right)^2 }} \\
v_g &= \frac{c_0}{\sqrt{\epsilon_r \mu_r}} \sqrt{1 - \left(\frac{f_c}{f}\right)^2 } = \frac{c_0}{\sqrt{\epsilon_r \mu_r}} \sqrt{1 - \left(\frac{\lambda_0}{\lambda_{c,v}}\right)^2}
\end{align*}

\item[Rundhohlleiter] \strut
\begin{center}
\begin{tabular}{ccc} \toprule
Bezeichnung & Symbol & Einheit \\ \midrule
Durchmesser & $D$ & \si{\meter} \\
Phasenmaß & $\beta_{mn}$ & \si{\per\meter} \\
Cutoff-Wellenlänge & $\lambda_{c,v}$ & \si{\meter} \\
\bottomrule
\end{tabular}
\end{center}
\begin{align*}
\lambda_{c,mn} &= \frac{\pi D}{p_{mn}} \\
\beta_{mn} &= \sqrt{k_0^2 - \left( \frac{p_{mn}}{D/2} \right)^2}
\end{align*}
\end{description}
\begin{center}
\begin{tabular}{cccccccc}
\toprule
$m$ & \multicolumn{3}{c}{$H_{mn}$} & & \multicolumn{3}{c}{$E_{mn}$} \\
\cmidrule(l){2-4}
\cmidrule(r){6-8}
 & $p_{m1}$ & $p_{m2}$ & $p_{m3}$ & & $p_{m1}$ & $p_{m2}$ & $p_{m3}$ \\
\midrule
0 & 3,832 & 7,016 & 10,174 & & 2,405 & 5,520 & 8,654 \\
1 & 1,841 & 5,331 & 8,536 & & 3,832 & 7,016 & 10,174 \\
2 & 3,054 & 6,706 & 9,970 & & 5,135 & 8,417 & 11,620 \\
\bottomrule
\end{tabular}
\end{center}