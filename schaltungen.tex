\begin{center}
\begin{tabular}{ccc} \toprule
Bezeichnung & Symbol & Einheit \\ \midrule
Güte des Elements & $Q$ & \\
\SI{3}{\dB}-Bandbreite & $B_\text{\SI{3}{\dB}}$ & \si{\hertz} \\
Transitfrequenz & $f_T$ & \si{\hertz} \\
Gleichstromverstärkung & $B$ & \\
Koppeldämpfung & $a_K$ & \si{\dB} \\
Durchlassdämpfung & $a_B$ & \si{\dB} \\
Richtdämpfung & $a_R$ & \si{\dB} \\
\bottomrule
\end{tabular}
\end{center}
\begin{description}
\item[Schwingkreise]
\begin{align*}
f_\text{res} = \frac{1}{2\pi \sqrt{LC}} \\
Q_K = \frac{f_\text{res}}{B_\text{\SI{3}{\dB}}} \\
Q_\text{Parallel} = R \sqrt{\frac{C}{L}}\\
Q_\text{Serie} = \frac{1}{R} \sqrt{\frac{L}{C}}
\end{align*}

% HFVO Folien 24-28
\item[Bipolar-Transistor] HF-Ersatzschaltbild nach Giacoletto
\begin{center}
\begin{circuitikz}[scale=1.35, every node/.style={scale=1}]
\node (B) at (0,0) {};
\node (B'1) at (1,0) {};
\node (B'2) at (1,-1) {};
\node (F1) at (2,0) {};
\node (F2) at (2,-1) {};
\node (G1) at (3,0) {};
\node (G2) at (3,-1) {};
\node (H1) at (4,0) {};
\node (H2) at (4,-1) {};
\node (I1) at (5,0) {};
\node (I2) at (5,-1) {};
\node (C) at (6,0) {};
\node (E1) at (0,-1) {};
\node (E2) at (6,-1) {};

\draw (B) to [R=$R_{BB'}$, o-*] (B'1);
%\par\ctikzset{voltage/distance from node/.initial=3}
\draw (B'1) to [R=$R_{B'E}$, v>=$U_{B'E}$, *-*] (B'2);
\draw (F1) to [C=$C_{B'E}$, *-*] (F2);
\draw (G1) to [R=$R_{CE}$, *-*] (G2);
\draw (H1) to [C=$C_{CE}$, *-*] (H2);
\draw (F1) to [R=$R_{B'C}$, *-*] (G1);
\draw (I1) to [csI=$S\cdot U_{B'E}$, *-*] (I2);
\draw (F1) --++(up:0.6) to [C=$C_{B'C}$] ++(right:1) to (G1);

\draw (B'1) to [short] (F1);
\draw (G1) to[short,-o] (C);
\draw (E1) to[short, o-o] (E2);

\draw (B) node [left] {B};
\draw (E1) node [left] {E};
\draw (E2) node [right] {E};
\draw (C) node [right] {C};
\draw (B'1) node [above] {B'};
\end{circuitikz}
\end{center}
\begin{align*}
S &\approx \frac{I_{E,gl}}{U_T}\\
C_{B',E} &\approx \frac{S}{2\pi f_T}\\
\frac{1}{R_{B',E}} &= G_{B',E} \approx \frac{S}{\beta_0}\\
\beta_0 &= \left.\frac{I_C}{I_B}\right\vert_{f \rightarrow 0} \approx \frac{I_{C,gl}}{I_{B,gl}} = B
\end{align*}
Bei $f_T$ gilt $\beta = 1$.

\item[Betriebsart A] AP liegt mitten in der Kennlinie, im Idealfall keine Nichtlinearitäten. $\eta \leq 0,5$.
\item[Betriebsart B] AP liegt am Rand der Kennlinie, Erzeugung von zu filternden Harmonischen. Keine Verluste bei fehlender Aussteuerung, $\eta \approx 0,786$.
\item[Betriebsart C] AP liegt außerhalb der Kennlinie, Erzeugung von zu filternden Harmonischen. Noch weniger Aussteuerung als B $\eta \approx 0,9$.
\item[Betriebsart D] AP liegt noch weiter außerhalb der Kennlinie, arbeitet nun als Schalter (Delta-Sigma). Durch Schaltvorgänge entstehen weitere Spektralanteile.

\item[Richtkoppler] \strut
\begin{center}
\begin{circuitikz}[scale=0.4, every node/.style={scale=1}]
%\node (1) at (0,0) {};
%\node (2) at (0,-1) {};
%\node (3) at (2,0) {};
%\node (4) at (2,-1) {};
%\node (F2) at (2,-1) {};
%\node (G1) at (3,0) {};
%\node (G2) at (3,-1) {};
%\node (H1) at (4,0) {};
%\node (H2) at (4,-1) {};
%\node (I1) at (5,0) {};
%\node (I2) at (5,-1) {};
%\node (C) at (6,0) {};
%\node (E1) at (0,-1) {};
%\node (E2) at (6,-1) {};
%
%\draw (B) to [R=$R_{BB'}$, o-*] (B'1);
%%\par\ctikzset{voltage/distance from node/.initial=3}
%\draw (B'1) to [R=$R_{B'E}$, v>=$U_{B'E}$, *-*] (B'2);
%\draw (F1) to [C=$C_{B'E}$, *-*] (F2);
%\draw (G1) to [R=$R_{CE}$, *-*] (G2);
%\draw (H1) to [C=$C_{CE}$, *-*] (H2);
%\draw (F1) to [R=$R_{B'C}$, *-*] (G1);
%\draw (I1) to [csI=$S\cdot U_{B'E}$, *-*] (I2);
%\draw (F1) --++(up:0.6) to [C=$C_{B'C}$] ++(right:1) to (G1);
%
%\draw (B'1) to [short] (F1);
%\draw (G1) to[short,-o] (C);
%\draw (E1) to[short, o-o] (E2);
%
%\draw (B) node [left] {B};
%\draw (E1) node [left] {E};
%\draw (E2) node [right] {E};
%\draw (C) node [right] {C};
%\draw (B'1) node [above] {B'};

\par\ctikzset{quadpoles/zweitor/height=.5}
\draw (0,0) node[zweitor] (G) {}
(G.A1) node[anchor=south] {1}
(G.A2) node[anchor=north] {2}
(G.B1) node[anchor=south] {3}
(G.B2) node[anchor=north] {4}
%(G.base) node[anchor=north] {\huge$\PKOLmatrix{S}$}
;

\draw (G.A1) to [short, *-, i_<=$P_1$] ++(-2,0) to [R, l_=$Z_L$] ++(-2,0) to [sV] ++(-2,0) to [short] ++(-0.5,0) node[rground] {};
\draw (G.A2) to [short, *-, i_=$P_2$] ++(-2,0) to [R, l_=$Z_L$] ++(-2,0) to [short] ++(-0.5,0) node[rground] {};
\draw (G.B1) to [short, *-, i=$P_3$] ++(2,0) to [R=$Z_L$] ++(2,0) to [short] ++(0.5,0) node[rground] {};
\draw (G.B2) to [short, *-, i=$P_4$] ++(2,0) to [R=$Z_L$] ++(2,0) to [short] ++(0.5,0) node[rground] {};

\draw (G.A1) to [short] (G.B1);
\draw (G.A2) to [short] (G.B2);

\draw[|->] (G.base) ++(0,-0.9) --+(135:1);
\draw[->] (G.base) ++(0,-0.9) --+(315:1);
%\draw (G.base) ++(0,-1.5) ++(-0.5,-0.5) --++(1,1);
\end{circuitikz}
\end{center}
\begin{align*}
a_K &= -10\log_\text{10} \frac{P_2}{P_1} = -20\log_\text{10} |S_{21}| \\
a_B &= -10\log_\text{10} \frac{P_3}{P_1} = -20\log_\text{10} |S_{31}| \\
a_R &= -10\log_\text{10} \frac{P_4}{P_2} =  -20\log_\text{10} \left\vert\frac{S_{41}}{S_{21}}\right\vert \qquad \text{möglichst groß, über \SI{40}{\dB}}
\end{align*}
Streuparameter eines idealen \SI{3}{\dB}-Kopplers:
\begin{equation*}
\PKOLmatrix{S} = \frac{1}{\sqrt{2}} \left[ \begin{matrix}
0 & 1 & -j & 0 \\ 
1 & 0 & 0 & -j \\ 
-j & 0 & 0 & 1 \\ 
0 & -j & 1 & 0 \\ 
\end{matrix} \right] \\
\end{equation*}

\item[Planare Resonatoren]
Bei kapazitiver Anordnung des Ringresonators wie hier entstehen Spannungsmaxima an den Koppelpunkten.
\begin{center}
\begin{tikzpicture}[scale=0.5, every node/.style={scale=1}]
\filldraw [pattern=north west lines, even odd rule] (0,0) circle(0.75) circle(1);
\filldraw [pattern=north west lines] (1.5,-0.5) rectangle(5,0.5);

\path [|-|,draw] (-1.2,1) -- ++(0,-2) node[midway,label=left:$D$] {};
\end{tikzpicture}
\end{center}
\begin{equation*}
f_\text{res} = n \cdot \frac{c_0}{D \pi \sqrt{\epsilon_{r,\text{eff}}}} \\
\end{equation*}

\item[Hohlraumresonatoren]
\begin{equation*}
f_{mnq} = \frac{\sqrt{ \left( \frac{m}{a} \right)^2 + \left( \frac{n}{b} \right)^2 + \left( \frac{q}{c} \right)^2}}{2\sqrt{\epsilon_0 \epsilon_r \mu_0 \mu_r}}
\end{equation*}
\end{description}