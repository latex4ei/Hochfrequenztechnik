%% Quellen
% http://www.siart.de/lehre/mehrtore.pdf
%
\begin{center}
\begin{tabular}{ccc} \toprule
Bezeichnung & Symbol & Einheit \\ \midrule
Hinlaufende Welle & $a$ & $\sqrt{\si{\watt}}$ \\
Rücklaufende Welle & $b$ & $\sqrt{\si{\watt}}$ \\
Bezugswiderstand & $Z_0$ & \si{\ohm} \\
Betriebsleistungsverstärkung & $g_T$ & \\
\bottomrule
\end{tabular}

\begin{tikzpicture}[scale=0.5, every node/.style={scale=1}]
\draw (0,0) node[zweitor] (G) {}
(G.A1) node[anchor=east] {}
(G.A2) node[anchor=east] {}
(G.B1) node[anchor=west] {}
(G.B2) node[anchor=west] {}
(G.base) node[anchor=north] {\huge$\PKOLmatrix{S}$}
;

\draw (G.A1) to [short, -o] ++(-2.5,0);
\draw (G.A2) to [short, -o] ++(-2.5,0);

\draw[->] (G.A1)++(-2.3,-0.4) --++ (0,-2) node[midway, right] {$U_r$};
\draw[->] (G.A1)++(-2.7,-0.4) --++ (0,-2) node[midway, left] {$U_h$};

\draw[<-] (G.A1)++(-2.3,0.4) --++ (2,0) node[midway, above] {$I_r$};
\draw[<-] (G.A1)++(-2.7,0.4) --++ (-2,0) node[midway, above] {$I_h$};

\draw[->, decorate,decoration={snake,post length=1mm,amplitude=1mm, segment length=1.5mm}] (G.A1)++(-1,-0.4) --++ (1.4,0) node[very near end, below] {$a$};
\draw[->, decorate,decoration={snake,post length=1mm,amplitude=1mm, segment length=1.5mm}] (G.A2)++(-1,0.4) ++(1.4,0) --++(-1.4,0) node[very near end, above] {$b$};
\end{tikzpicture}
\end{center}

\begin{description}
\item[Torgrößen]
\renewcommand{\arraystretch}{1.6}
\begin{equation*}
\begin{array}{cc}
U = U_h + U_r &  I = I_h - I_r = \frac{U_h}{Z_0} - \frac{U_r}{Z_0}
\end{array}
\end{equation*}
\begin{align*}
a &= \frac{U + I Z_0}{2 \sqrt{Z_0}} = \frac{U_h}{\sqrt{Z_0}} = I_h \sqrt{Z_0} \\
b &= \frac{U - I Z_0}{2 \sqrt{Z_0}} = \frac{U_r}{\sqrt{Z_0}} = I_r \sqrt{Z_0}
\end{align*}
\begin{equation*}
\begin{array}{cc}
U = \sqrt{Z_0} (a + b) & I = \frac{1}{\sqrt{Z_0}} (a - b) \\
u = \frac{U}{\sqrt{Z_0}} = (a + b) & i = I \sqrt{Z_0} = (a - b) \\
a = \frac{1}{2} (u+i) & b = \frac{1}{2} (u-i)
\end{array}
\end{equation*}
\begin{align*}
S &= \frac{1}{2} U I^* = \frac{1}{2}\left( |a|^2 - |b|^2 + ba^* - b^*a \right) \\
P_W &= \Re\{S\} = \frac{1}{2} |a|^2 - \frac{1}{2} |b|^2
\end{align*}

\item[Abgegebene Generatorleistung]
\begin{equation*}
P_G = |a_G|^2 \frac{1-|\Gamma_1|^2}{|1-\Gamma_E\Gamma_1|^2}
\end{equation*}

\item[Maximale verfügbare Generatorleistung] Innenwiderstand $R_i = \Re\left\{Z_i\right\}$
\begin{equation*}
P_\text{max} = \frac{|U_0|^2}{8R_i} = \frac{|a_G|^2}{1-|\Gamma_1|^2}
\end{equation*}

\item[Prinzip der durchgehenden Wirkleistung] Gültig bei verlustlosen Zweitoren. Wenn ein Zweitor verlustbehaftet ist, wird es durch das Vorziehen der verlustbehafteten Komponenten wieder verlustfrei.
\begin{align*}
\left\vert \frac{U_n}{U_1} \right\vert = \sqrt{\frac{G_1}{G_n}} \\
\left\vert \frac{I_n}{I_1} \right\vert = \sqrt{\frac{R_1}{R_n}} \\
\end{align*}
Da $R \neq \frac{1}{G}$ erfolgt die Umrechnung zwischen $R$ und $G$ durch die folgenden Näherungsformeln:
\begin{equation*}
\begin{array}{ccc}
G \gg |B| & R \approx \frac{1}{G} & X \approx - \frac{B}{G^2} \\
|B| \gg G & R \approx \frac{G}{B^2} & X \approx - \frac{1}{B} \\
R \gg |X| & G \approx \frac{1}{R} & B \approx - \frac{X}{R^2} \\
|X| \gg R & G \approx \frac{R}{X^2} & B \approx - \frac{1}{X} \\
\end{array}
\end{equation*}
\renewcommand{\arraystretch}{1}

\item[Beschaltete Zweitore] \strut
\begin{center}
\begin{circuitikz}[scale=1, every node/.style={scale=1}]
%\node [contact,label=above:1] (contact 1) at (0,1.5) {};
%\node [contact,label=below:1'] (contact 1') at (0,0) {};
%\node [contact,label=above:2] (contact 2) at (3,1.5) {};
%\node [contact,label=below:2'] (contact 2') at (3,0) {};
%
%\draw (contact 1) to [current direction'={very near start}, resistor={info=$Z_i$}] ++(left:2)
%to [voltage source={midway, direction info, info'=$U_0$}] ++(down:1.5)
%to (contact 1');
%
%\draw (0.5,-0.5) rectangle(2.5,2);
%
%\draw (contact 1) -- ++(0.5,0);
%\draw (contact 1') -- ++(0.5,0);
%\draw (contact 2) -- ++(-0.5,0);
%\draw (contact 2') -- ++(-0.5,0);
%
%\node [text centered, font=\huge] at (1.5,0.75) {$\underline{S}$};
%
%\draw (contact 2) -- ++(right:0.7) to [resistor={info=$Z_\text{Last}$}] ++(down:1.5)
%to (contact 2');
%
\draw [->] (G.A2)  ++(-0.2,-0.5) node[anchor=north] {$\Gamma_E$} -- ++(0,1.25) -- ++(-0.2,0);
\draw [->] (G.A2)  ++(0.2,-0.5) node[anchor=north] {$\Gamma_1$} -- ++(0,1.25) -- ++(0.2,0);
%\node[label=left:$\Gamma_E$] at (-0.2,-0.5) {};
%\node[label=below:$\Gamma_1$] at (0.2,-0.5) {};
%
\draw [->] (G.B2)  ++(-0.2,-0.5) node[anchor=north] {$\Gamma_2$} -- ++(0,1.25) -- ++(-0.2,0);
\draw [->] (G.B2)  ++(0.2,-0.5) node[anchor=north west] {$\Gamma_\text{Last}$} -- ++(0,1.25) -- ++(0.2,0);
%\node[label=below:$\Gamma_2$] at (2.8,-0.5) {};
%\node[label=right:$\Gamma_\text{Last}$] at (3.2,-0.5) {};
%
%\draw (contact 2) ++(0,-0.4) node {$Z_L$};
%\draw (contact 1) ++(0,-0.4) node {$Z_L$};

\draw (0,0) node[zweitor] (G) {}
(G.A1) node[anchor=south] {1}
(G.A1) node[anchor=north, yshift=-0.2cm] {$Z_L$}
(G.A2) node[anchor=north] {1'}
(G.B1) node[anchor=south] {2}
(G.B1) node[anchor=north, yshift=-0.2cm] {$Z_L$}
(G.B2) node[anchor=north] {2'}
(G.base) node[anchor=north] {\huge$\PKOLmatrix{S}$}
;

\draw (G.A1) to [R=$Z_i$, *-] ++(-1.4,0);
\draw (G.A2) to [short, *-] ++(-1.4,0);

\draw (G.A1)++(-1.4,0) to [sV_=$U_0$] ($(G.A2)+(-1.4,0)$);

\draw (G.B1) to [short, *-] ++(0.7,0);
\draw (G.B2) to [short, *-] ++(0.7,0);
\draw ($(G.B1)+(0.7,0)$) to [R=$Z_\text{Last}$] ($(G.B2)+(0.7,0)$);
\end{circuitikz}
\end{center}
\begin{align*}
\Gamma_1 &= S_{11} + \frac{S_{12} S_{21} \Gamma_\text{Last}}{1 - S_{22} \Gamma_\text{Last}} \\
\Gamma_\text{Last} &= \frac{Z_\text{Last} - Z_L}{Z_\text{Last} + Z_L} \\
\Gamma_2 &= S_{22} + \frac{S_{12} S_{21} \Gamma_E}{1 - S_{11} \Gamma_E} \\
\Gamma_E &= \frac{Z_i - Z_L}{Z_i + Z_L} \\
Z_1 &= Z_0 \frac{1+\Gamma_1 }{1-\Gamma_1} \\
g_T &= \frac{P_2}{P_1} = \frac{ |S_{21}|^2 (1-|\Gamma_E|^2) (1-|\Gamma_\text{Last}|^2) }{\left\vert (1-S_{11}\Gamma_E)(1-S_{22}\Gamma_\text{Last}) - S_{12}S_{21} \Gamma_\text{Last}\Gamma_E \right\vert ^2}
\end{align*}
\end{description}