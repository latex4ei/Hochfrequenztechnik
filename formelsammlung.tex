\documentclass[DIV=calc, fontsize=7pt, paper=a4, twocolumn]{scrartcl}

\usepackage[ngerman]{babel}		
\usepackage[T1]{fontenc}
\usepackage[utf8]{inputenc}	
\usepackage{amsmath}
\usepackage{amssymb}
\usepackage{babel}
\usepackage{siunitx}
\usepackage{graphicx}
\usepackage{booktabs-de}	
\usepackage{tikz}
\usetikzlibrary{arrows}

\newcommand{\mbeq}{\overset{!}{=}}

\sisetup{
%list-final-separator = { \translate{und} },
%range-phrase = { \translate{bis} },
per-mode = fraction,
locale=DE,
exponent-product = \cdot
}

\DeclareSIUnit{\rad}{rad}
%\DeclareSIUnit{\dBm}{dBm}

\title{Formelsammlung Hochfrequenztechnik}
\author{Patrick Kolesa, studium@pkol.de}
\date{}

\begin{document}
\maketitle
\section*{Konstanten}
\begin{center}
\begin{tabular}{ccc} \toprule
Bezeichnung & Symbol & Wert \\ \midrule
Lichtgeschwindigkeit im Vakuum & $c_0$ & \SI{3e8}{\meter \per \second} \\
Elektrische Feldkonstante & $\epsilon_0$ & \SI{4\pi}{\volt\second\per\ampere\per\meter}\\
Magnetische Feldkonstante & $\mu_0$ & \SI{8,85e-12}{\ampere\second\per\volt\per\meter}\\
Freiraumwellenwiderstand & $Z_{F0}$ & $\SI{120\pi}{\ohm} \approx \SI{377}{\ohm}$ \\ 
\bottomrule
\end{tabular}
\end{center}

\section*{Nützliche mathematische Formeln}
\paragraph{Polarkoordinaten $\longmapsto$ Re-Im}
\begin{equation*}
Z = \text{Re} + j\text{Im} \textrm{ mit Re } = \vert r \vert \cos \varphi \textrm{ , Im } = \vert r \vert \sin \varphi
\end{equation*}
\paragraph{Re-Im $\longmapsto$ Polarkoordinaten}
\begin{equation*}
Z = r \exp(j \varphi) \textrm{ mit } r = \sqrt{\textrm{Re}^2 + \textrm{Im}^2} \text{ , } \varphi = \left\{ \begin{array}{cc}
+\arccos \frac{\text{Re}}{r} & \textrm{Im} \geq 0 \\ 
-\arccos \frac{\text{Re}}{r} & \textrm{Im} < 0
\end{array} \right.
\end{equation*}

\section*{Hochfrequente Schaltungen}
\begin{center}
\begin{tabular}{ccc} \toprule
Bezeichnung & Symbol & Einheit \\ \midrule
Reflexionsfaktor & $\Gamma$ &  \\
Durchgangsdämpfung & $a_B$ & \si{\dB} \\
\bottomrule
\end{tabular}
\end{center}
%\begin{center}
%\begin{tikzpicture}[scale=0.3, every node/.style={scale=0.8}]
%\node[text centered, font=\huge] at (0,0) {$\underline{S}$};

\draw (-2,-2) rectangle(2,2);

\draw[-o] (2,1.1) --(4,1.1) node[anchor=south] {2};
\draw (5,0.6) node[anchor=east,label=right:$a_2$]{};
\draw [->] (5,0.6)--(3.5,0.6);
%\draw (4,1.1) node[shape=circle,anchor=west,draw,label=above:2]{};

\draw (-2,1.1) --(-4,1.1);
\draw (-5,0.6) node[anchor=west,label=left:$a_1$]{};
\draw [->] (-5,0.6)--(-3.5,0.6);
\draw (-4,1.1) node[shape=circle,anchor=east,draw,label=above:1]{};

\draw (2,-1.1) --(4,-1.1);
\draw (5,-0.6) node[anchor=east,label=right:$b_2$]{};
\draw [->] (3.5,-0.6)--(5,-0.6);
\draw (4,-1.1) node[shape=circle,anchor=west,draw,label=below:2']{};

\draw (-2,-1.1) --(-4,-1.1);
\draw (-5,-0.6) node[anchor=west,label=left:$b_1$]{};
\draw [<-] (-5,-0.6)--(-3.5,-0.6);
\draw (-4,-1.1) node[shape=circle,anchor=east,draw,label=below:1']{};

\node at (0,-3){$a,b = \left[ \sqrt{\si{\watt}} \right]$};
%\end{tikzpicture}
%\end{center}

\begin{center}
\begin{tikzpicture}[scale=0.6, every node/.style={scale=1}]
\node (contact 1) at (0,1.5) {};
\node (contact 1s) at (0,2) {};
\node (contact 1') at (0,0) {};
\node (contact 1's) at (0,-0.5) {};

\node (contact 2) at (5.5,1.5) {};
\node (contact 2s) at (5.5,2) {};
\node (contact 2') at (5.5,0) {};
\node (contact 2's) at (5.5,-0.5) {};


%\draw (contact 1) to [current direction'={very near start}, resistor={info=$Z_i$}] ++(left:2) to [voltage source={midway, direction info, info'=$U_0$}] ++(down:1.5) to (contact 1');

\draw (contact 1) node [left] {Tor 1};
\draw (contact 2') node [right] {Tor 2};

\draw[thick] (2,-0.5) rectangle ++(1.5,2.5) node[midway,text centered, font=\huge] {$\PKOLmatrix{S}$};

\draw[o-] (contact 1) -- ++(2,0);
\draw[o-] (contact 1') -- ++(2,0);

\draw[o-] (contact 2) -- ++(-2,0);
\draw[o-] (contact 2') -- ++(-2,0);

\draw[->] (contact 1s) .. controls ++(2.75,0.5) .. (contact 2s) node [right] {$S_{21}$};
\draw[<-] (contact 1's) node [left] {$S_{12}$} .. controls ++(2.75,-0.5) .. (contact 2's);


\draw[->] (0,1.25) .. controls (2,0.75) .. (0,0.25) node [left] {$S_{11}$};
\draw[<-] (5.5,1.25) node [right] {$S_{22}$} .. controls (3.5,0.75) .. (5.5,0.25);

%\draw[->] (contact 1)++(0.2,-0.2) --++ (0,-1.1) node[midway, right] {$U_r$};
%
%
%\draw[<-] (contact 1)++(0.2,0.2) --++ (1,0) node[midway, above] {$I_r$};
%\draw[<-] (contact 1)++(-0.2,0.2) --++ (-1,0) node[midway, above] {$I_h$};
%
%
%\draw[->, decorate,decoration={snake,post length=1.4mm,amplitude=1mm, segment length=3mm}] (contact 1)++(0.7,-0.2) --++ (1,0) node[very near end, below] {$a$};
%\draw[->, decorate,decoration={snake,post length=1.4mm,amplitude=1mm, segment length=3mm}] (contact 1')++(0.7,0.2) ++ (1,0)--++(-1,0) node[very near end, above] {$b$};

\end{tikzpicture}
\end{center}

\begin{align*}
\left( \begin{array}{c}
b_1 \\ 
b_2
\end{array}
\right) &= \left[
\begin{array}{cc}
S_{11} & S_{12} \\ 
S_{21} & S_{22}
\end{array}
\right] \cdot \left(
\begin{array}{c}
a_1 \\ 
a_2
\end{array} \right) \\
a_k &= \frac{1}{2\sqrt{2Z_0}}\left( U_k + Z_0 I_k\right) \\
b_k &= \frac{1}{2\sqrt{2Z_0}}\left( U_k - Z_0 I_k\right)
\end{align*}
\renewcommand{\arraystretch}{1.6}
\begin{align*}
\begin{array}{cc}
S_{11} = \left.\frac{b_1}{a_1} \right\vert_{a_2 = 0} & S_{12} = \left.\frac{b_1}{a_2} \right\vert_{a_1 = 0} \\ 
S_{21} = \left.\frac{b_2}{a_1} \right\vert_{a_2 = 0} & S_{22} = \left.\frac{b_2}{a_2} \right\vert_{a_1 = 0}
\end{array}
\end{align*}
\renewcommand{\arraystretch}{1}

\begin{description}
\item[Umrechnung in \si{\dB}]
\begin{equation*}
\begin{array}{cc}
\Gamma_{\si{\dB}_1} = 20\log\frac{1}{\vert S_{11} \vert}
 %= 10 \log \frac{1}{\vert S_{11} \vert^2}
& a_{B2} = 20\log\frac{1}{\vert S_{12} \vert}
 %= 10 \log \frac{1}{\vert S_{12} \vert^2}
 \\ 
a_{B1} = 20\log\frac{1}{\vert S_{21} \vert}
 %= 10 \log \frac{1}{\vert S_{21} \vert^2}
 & \Gamma_{\si{\dB}_2} = 20\log\frac{1}{\vert S_{22} \vert}
 %= 10 \log \frac{1}{\vert S_{22} \vert^2}
\end{array}
\end{equation*}

\item[Prozentuale Verlustleistung bei Fehlanpassung]
\begin{equation*}
P_v = |\Gamma|^2
\end{equation*}

\item[Anpassung] Diagonale der S-Matrix ist Null.
\begin{equation*}
S_{ii} = 0 \equiv -\infty\,\si{\dB}
\end{equation*}

\item[Reziprozität] S-Matrix ist symmetrisch.
\begin{equation*}
S_{ji} = S_{ij}
\end{equation*}

\item[Verlustlosigkeit] S-Matrix ist unitär.
\begin{equation*}
\underline{S}^T \cdot \underline{S}^* = I
\end{equation*}

\item[Rückwirkungsfrei] Keine Übertragung von Tor 2 zu Tor 1.
\begin{equation*}
S_{12} = 0
\end{equation*}

\item[Einschränkung] Es gibt kein verlustloses, angepasstes und reziprokes Dreitor.

\item[Häufige S-Matrizen] 
\begin{align*}
\text{Verlustlose Leitung: }& \left[ \begin{matrix}
0 & e^{-j\beta\Delta z} \\ 
e^{-j\beta\Delta z} & 0
\end{matrix} \right] \\
%\text{Rückwirkungsfreier Verstärker: }& \left[ \begin{matrix}
%0 & 0 \\ 
%1,7e^{-j160^\circ} & 0,2e^{j42^\circ}
%\end{matrix} \right] \\
\end{align*}
\end{description}
\input{leitungstheorie}
\begin{center}
\begin{tabular}{ccc} \toprule
Bezeichnung & Symbol & Einheit \\ \midrule
Leitungswellenwiderstand & $Z_L$ &  \si{\ohm}\\
Bezugswellenwiderstand & $Z_0$ &  \si{\ohm}\\
Anpasspunkt (normiert) & $z_0$ &  \\
Reflexionsfaktor & $\Gamma$ &  \\
Voltage Standing Wave Ratio & VSWR &  \\
\bottomrule
\end{tabular}
\end{center}

\subsection*{Allgemein}
\begin{description}
\item[Normierung] Alle Impedanzen $Z$ werden im SD normiert als $z$ dargestellt.
\begin{equation*}
z = \frac{Z}{Z_0}
\end{equation*}
\end{description}

\subsection*{Reflexionsfaktor \texorpdfstring{$\Gamma$}{r}}
\begin{description}
\item[Kurzschluss] Für geringere Leitungslänge ist das Verhalten induktiv.
\begin{equation*}
\Gamma = -1
\end{equation*}

\item[Leerlauf] Für geringere Leitungslänge ist das Verhalten kapazitiv.
\begin{equation*}
\Gamma = 1
\end{equation*}

\item[Reflexionsfreier Abschluss] Anpasspunkt bei $z_0 = 1+j0$
\begin{equation*}
\Gamma = 0
\end{equation*}

\item[Impedanz einer abgeschlossenen Leitung]
\begin{equation*}
\Gamma = \frac{Z_L - Z_0}{Z_L + Z_0}
\end{equation*}
\end{description}

\subsection*{Anpassprobleme}
\highlight{schritte}{Schritte zur Anpassung von $Z_\text{Last}$ (nur konzentrierte Elemente)}{
Es wird von der Impedanzebene ausgegangen.
\begin{enumerate}
\item Anpasskreis für Admittanzebene zeichnen, Mittelpunkt bei $z=0,35$.
\item Es kann angepasst werden, falls $Z_\text{Last}$ auf einen Anpasskreis gebracht werden kann. Dazu den Linien mit $\Re=\const$ folgen und entweder Kapazität oder Induktivität auswählen.
\begin{itemize}
\item Man sollte den Anpasskreis der Admittanzebene erreichen.
\item Falls nicht, so wird $Z_\text{Last}$ am Anpasspunkt gespiegelt und es wird nochmal in der Admittanzebene versucht. Anschließend wieder zurückspiegeln.
\end{itemize}
\item Am Anpasskreis wird jetzt eine Parallel- bzw. Reihenschaltung eines weiteren Bauelements vorgenommen, so dass man im Anpasspunkt landet.
\item %Das Ablesen der Bauteilwerte erfolgt nur in der Impedanzebene!
Das Ablesen der Bauteilwerte erfolgt für Reihenschaltung in der Impedanzebene, ananlog für Parallelschaltung in der Admittanzebene. Dazu die Differenz des Imaginärteils zwischen den jeweiligen Punkten bilden und in die unteren Gleichungen einsetzen.
\end{enumerate}
}
\highlight{schritte}{Parallele Stichleitung zur Anpassung}{
\begin{enumerate}
\item $Z_\text{Last}$ durch Spiegelung am Anpasspunkt zu $Y_\text{Last}$ transformieren
\item Imaginärteil ablesen, muss durch Stichleitung erzeugt werden
\item Vom Anpasspunkt gerade Linie zum Imaginärteil am Rand zeichnen
\item $\frac{l}{\lambda}$ ablesen
\end{enumerate}
}
\begin{description}
\item[Leistungsanpassung] Wenn $Z_i$ Generatorimpedanz, dann muss man durch geeignete Anpassung zu $Z^*_i$ gelangen.
\item[$\lambda / 4$-Transformation] Jeder Widerstand (reelle Impedanz) $Z_\text{Last}$ kann mit einer $\lambda / 4$-langen Leitung mit Wellenwiderstand $Z_{\lambda /4}$ über
\begin{equation*}
Z_{\lambda /4} = \sqrt{Z_0 Z_\text{Last}}
\end{equation*}
in den Anpasspunkt $Z_L$ transformiert werden.
\end{description}

\subsection*{Elementwerte bestimmen}
\begin{description}
\item[Induktivität] Imaginärteil-Linien zum Rand folgen
\begin{align*}
L_S &= \frac{Z_0 \Delta x}{2\pi f} \\
L_P &= \frac{Z_0}{2\pi f \Delta b}
\end{align*}

\item[Kapazität] Imaginärteil-Linien zum Rand folgen
\begin{align*}
C_S &= \frac{1}{2\pi f Z_0 \Delta x} \\
C_P &= \frac{\Delta b}{2\pi f Z_0}
\end{align*}

\item[Leitungslänge] Gerade von Mittelpunkt zum Rand
\begin{align*}
l &= \frac{c_0}{f} \frac{l}{\lambda}
\end{align*}
\end{description}
\begin{center}
\begin{tabular}{ccc} \toprule
Bezeichnung & Symbol & Einheit \\ \midrule
Güte des Elements & $Q$ & \\
Verlustwinkel & $\delta$ & \\
Scheinbare Induktivität & $L^*$ & \si{\henry} \\
Scheinbare Kapazität & $C^*$ & \si{\farad} \\
\bottomrule
\end{tabular}
\end{center}

\begin{description}
\item[Widerstand] Vereinfachend für kleine $R$ nur $L_s$ und für große $R$ nur $C_p$.
\begin{center}
\begin{circuitikz}[scale=0.5, every node/.style={scale=1}]
\node (C1) at (0,1.3) {};
\node (C2) at (5,1.3) {};
\node (L1) at (0,0) {};
\node (L2) at (2.5,0) {};
\node (R2) at (5,0) {};

\draw (C1) to [capacitor=$C_p$] (C2);
\draw (L1) to [american inductor=$L_s$] (L2) to [R=$R$] (R2);

\draw (C1) to [short, -*] (L1);
\draw (C2) to [short, -*] (R2);

\draw (L1) to [short, -o] ++(-0.75,0);
\draw (R2) to [short, -o] ++(0.75,0);
\end{circuitikz}
\end{center}
\begin{align*}
Z_\text{ideal} &= j \omega L \\
Z_\text{real} &= \frac{j\omega L}{1 - \omega^2 LC_p} \\
Q_L &= \frac{\omega L}{R_s} = \frac{1}{\tan \delta_L} \\
R_s &= R_\text{Fe} + R_\text{Cu} = \omega L \tan\delta_\mu + R_\text{Cu} \\
L^* &= \frac{L}{1-\omega^2 C_p L} > L
\end{align*}

\item[Induktivität] Wird $\mu_r$ erhöht, so steigt $L$. Überhalb der Resonanzfrequenz verhält sich die reale Induktivität kapazitiv.
\begin{center}
\begin{circuitikz}[scale=0.5, every node/.style={scale=1}]
\input{bilder/induktivitaet1.tex}
\end{circuitikz}
\end{center}
\begin{align*}
Z_\text{ideal} &= j \omega L \\
Z_\text{real} &= \frac{j\omega L}{1 - \omega^2 LC_p} \\
Q_L &= \frac{\omega L}{R_s} = \frac{1}{\omega L G_p} = \frac{1}{\tan \delta_L} \\
R_s &= R_\text{Fe} + R_\text{Cu} = \omega L \tan\delta_\mu + R_\text{Cu} \\
L^* &= \frac{L}{1-\omega^2 C_p L} > L
\end{align*}

\item[Induktivität durch kurzgeschlossene Leitung]
\begin{align*}
Z &= jX = j Z_L \tan( \beta l)
\intertext{falls $l \ll \lambda_0$}
X &= \omega L' l \\
\end{align*}

\item[Kapazität] Überhalb der Resonanzfrequenz verhält sich die reale Kapazität induktiv.
\begin{center}
\begin{circuitikz}[scale=0.5, every node/.style={scale=1}]
%\node [contact] (contact 1) at (0,2) {};
%\node [contact] (contact 2) at (6,2) {};
%\node [contact] (contact 3) at (3,2) {};
%
%\draw (contact 1) to [inductor={info'=$L_s$, midway}] (contact 3) to [capacitor={info'=$C$,midway}] (contact 2);
%\draw (contact 3) --++(up:1) to [resistor={info=$G_p$}] ++(right:3) to (contact 2);

%
\node (C1) at (2.5,1.5) {};
\node (C2) at (5,1.5) {};
\node (L1) at (0,0) {};
\node (L2) at (2.5,0) {};
\node (R2) at (5,0) {};

\draw (C1) to [R=$G_p$] (C2);
\draw (L1) to [american inductor=$L_s$] (L2) to [C=$C$] (R2);

\draw (C1) to [short, -*] (L2);
\draw (C2) to [short, -*] (R2);

\draw (L1) to [short, -o] ++(-0.75,0);
\draw (R2) to [short, -o] ++(0.75,0);
\end{circuitikz}
\end{center}
\begin{align*}
Y_\text{ideal} &= j \omega C \\
Y_\text{real} &= \frac{j \omega C}{1 - \omega^2 L_sC} \\
Q_C &= \frac{\omega C}{G_p} = \frac{1}{\omega C R_s} = \frac{1}{\tan \delta_C} \\
\tan \delta_C &= \frac{\epsilon_r''}{\epsilon_r'} \text{ für homogenes Dielektrikum}\\
C^* &= \frac{C}{1 - \omega^2 L_s C} > C \\
\end{align*}

\item[Kapazität durch leerlaufende Leitung]
\begin{align*}
Y &= jB = j \frac{1}{Z_L} \tan( \beta l)
\intertext{falls $l \ll \lambda_0$}
B &= \omega C' l \\
\end{align*}

% HFVO Folie 12
\item[Schottky-Diode] Metall-Halbleiter-Übergang, fehlende Minoritätsträger ermöglichen Nutzung für höchste Frequenzen. Anwedungen: Gleichrichter, Detektoren, Mischer.

% HFVO Folie 14
\item[Kapazitätsdiode] Betrieb in Sperrrichtung. Hohe Sperrspannung bedeutet niedrige Kapazität und umgekehrt. Anwendungen: Abstimmung von Schwingkreisen, Filtern.

% HFVO Folie 15
\item[Tunneldiode] Kennlinie besitzt wegen Tunneleffekt Abschnitt mit negativem Widerstand. Kurzschlussstabil. Anwendung: Breitbandige Verstärkung von Signalen.

% HFVO Folie 16
\item[Impatt-Diode] Kennlinie besitzt durch Laufzeitverzerrungen Abschnitt mit negativem Widerstand. Anwendung: Verstärkung von Signalen.

% HFVO Folie 20
\item[PIN-Diode] Erzeugt mittels eines Gleichstroms einen Wechselstromwiderstand. Anwendungen: HF-Schalter, Wellenwiderstandsanpassung.

% HFVO Folie 23
\item[Gunn-Element] Durch Gunn-Effekt entsteht ein Kennlinienabschnitt mit negativem Widerstand. Anwendung: Oszillator.

\end{description}

\section*{Maxwellsche Gleichungen}

\section*{Ebene Wellen}
\paragraph{Skineffekt} Bei $\Delta z = 1\delta$: Feldstärke $E_0$ fällt auf $0,37 E_0$ ab.
\begin{equation*}
\delta = \sqrt{\frac{2}{\omega \mu \kappa}}
\end{equation*}

\section*{Wellenausbreitung}

\section*{Antennen}

\end{document}