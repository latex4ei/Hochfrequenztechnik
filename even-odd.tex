%\begin{center}
%\begin{tabular}{ccc} \toprule
%Bezeichnung & Symbol & Einheit \\ \midrule
%Wellenwiderstand & $Z_F$ & \si{\ohm} \\
%Brewsterwinkel (nur parallel) & $\theta_B$ &  \\
%Wellenzahl & k & \\
%Eindringtiefe & $\delta$ & \si{\meter} \\
%Poynting-Vektor & $\vec{S}$ & $\si{\watt\per\meter^2}$ \\
%\bottomrule
%\end{tabular}
%\end{center}

\begin{description}
\item[Allgemein]
\begin{align*}
U &= U_e + U_o \\
\end{align*}

\item[Gleichtakt (Even)] Gleichphasige Anregung an beiden Toren führt zur Aufhebung der Ströme in der Symmetrieebene.
\begin{center}
\begin{circuitikz}[scale=1, every node/.style={scale=1}]
\node (A1) at (0,0) {};
\node (A2) at (0,-1.4) {};
\node (M1) at (1,0) {};
\node (M2) at (1,-1.4) {};
\node (B1) at (2,0) {};
\node (B2) at (2,-1.4) {};
\node (C1) at (5.5,0) {};
\node (C2) at (5.5,-1.4) {};
\node (D1) at (6.5,0) {};
\node (D2) at (6.5,-1.4) {};

\draw (A1) to [R=$Z_i$, *-] ++(-1.4,0);
\draw (A2) to [short, *-] ++(-1.4,0);
\draw (A1)++(-1.4,0) to [sV=$U_0$] ($(A2)+(-1.4,0)$);

\draw (B1) to [R, l_=$Z_i$, *-] ++(1.4,0);
\draw (B2) to [short, *-] ++(1.4,0);
\draw (B1) ++(1.4,0) to [sV_=$U_0$] ($(B2)+(1.4,0)$);

\draw (A1) to [short, i^=$I$] (M1) to [short, i^<=$I$] (B1);
\draw (A2) to [short] (M2) to [short] (B2);


\draw (C1) to [R, l_=$Z_i$, *-] ++(-1.4,0);
\draw (C2) to [short, *-] ++(-1.4,0);
\draw (C1) ++(-1.4,0) to [sV=$U_0$] ($(C2)+(-1.4,0)$);

\draw (C1) to [short] (D1);
\draw (C2) to [short] (D2);

\draw (D1) to [open, v^>=$U$] (D2);
\draw (C1) to [open, v^>=$U_e$] (C2);
\draw (B1) to [open, v_>=$U_e$] (B2);

\draw[dashed, very thin] (M1) ++(0,0.3) --++(0,-2);
\draw (A1) to [open, v^>=$U_e$] (A2);
\draw (M1) to [open, v^>=$U$] (M2);
\draw (M1) to [open, v_>=$U$] (M2);
\end{circuitikz}
\end{center}

\item[Gegentakt (Odd)] Gegenphasige Anregung an beiden Toren führt zur Aufhebung der Spannungen in der Symmetrieebene.
\begin{center}
\begin{circuitikz}[scale=1, every node/.style={scale=1}]
\node (A1) at (0,0) {};
\node (A2) at (0,-1.4) {};
\node (M1) at (1,0) {};
\node (M2) at (1,-1.4) {};
\node (B1) at (2,0) {};
\node (B2) at (2,-1.4) {};
\node (C1) at (5.5,0) {};
\node (C2) at (5.5,-1.4) {};
\node (D1) at (6.5,0) {};
\node (D2) at (6.5,-1.4) {};

\draw (A1) to [R=$Z_i$, *-] ++(-1.4,0);
\draw (A2) to [short, *-] ++(-1.4,0);
\draw (A1)++(-1.4,0) to [sV=$U_0$] ($(A2)+(-1.4,0)$);

\draw (B1) to [R, l_=$Z_i$, *-] ++(1.4,0);
\draw (B2) to [short, *-] ++(1.4,0);
\draw (B1) ++(1.4,0) to [sV_<=$U_0$] ($(B2)+(1.4,0)$);

\draw (A1) to [short, i^=$I$] (M1) to [short, i^>=$I$] (B1);
\draw (A2) to [short] (M2) to [short] (B2);


\draw (C1) to [R, l_=$Z_i$, *-] ++(-1.4,0);
\draw (C2) to [short, *-] ++(-1.4,0);
\draw (C1) ++(-1.4,0) to [sV=$U_0$] ($(C2)+(-1.4,0)$);

\draw (C1) to [short] (D1);
\draw (C2) to [short] (D2);

\draw (D1) to [short, i^>=$I$] (D2);
\draw (C1) to [open, v^>=$U_o$] (C2);
\draw (B1) to [open, v_<=$U_o$] (B2);

\draw[dashed, very thin] (M1) ++(0,0.3) --++(0,-2);
\draw (A1) to [open, v^>=$U_o$] (A2);
\draw (M1) to [open, v^<=$U$] (M2);
\draw (M1) to [open, v_>=$U$] (M2);
\end{circuitikz}
\end{center}

\item[Bauteile in Symmetrieebene] Befindet sich z.\,B. eine Impedanz $Z$ genau in der Ebene, so wird für die Analyse der Wert $2Z$ verwendet. Denn sobald man zwei Impedanzen mit Wert $2Z$ parallelschaltet, ergibt sich wieder $Z$.
\end{description}